% DO NOT put a magic command here as it messes with output directory when trying the custom build code
% !TeX root = ../main.tex



% So what do I change between report, thesis and presentation?
% change the Bib type, the geometry, the dependencies w.r.t thesis.cls, section numbering, thesis_cls mise en page, graphicx folder ?
% one day will try to make a global header and call it GOD_mode :p
% ---------------- Following command redefines sections as the mother ------------------
\renewcommand\thesection{\arabic{section}}

% --------------------------------------------------------------------------------------
% --------------------------------------- Packages -------------------------------------
% --------------------------------------------------------------------------------------
\usepackage{alltt}
\usepackage{amscd}
\usepackage{amsfonts}
\usepackage{amsmath}
\numberwithin{equation}{section}
\usepackage{amssymb}
\usepackage{amsthm} 
\usepackage[english]{babel}
\usepackage{booktabs}
\usepackage{bbm}
\usepackage{bibentry}
\usepackage[makeroom]{cancel}
\usepackage{caption}
\usepackage{cleveref}
    \crefformat{section}{\S#2#1#3}
    \crefformat{subsection}{\S#2#1#3}
    \crefformat{subsubsection}{\S#2#1#3}
    \crefrangeformat{section}{\S\S#3#1#4 to~#5#2#6}
    \crefmultiformat{section}{\S\S#2#1#3}{ and~#2#1#3}{, #2#1#3}{ and~#2#1#3}

\usepackage{changepage}
\usepackage{enumitem}
\usepackage{etoolbox}
\usepackage{faktor}
\usepackage{fancyhdr}
\usepackage[T1]{fontenc}

\usepackage{geometry}
    \addtolength{\oddsidemargin}{+.5in} % left side
    \addtolength{\evensidemargin}{+.5in} % right side
    \addtolength{\textwidth}{1in}
    \addtolength{\topmargin}{+.875in}
    \addtolength{\textheight}{1.5in}

\usepackage{graphicx}
\graphicspath{ {./images/} } %wrt main.tex i think

\usepackage{hyperref}
\hypersetup{urlcolor=blue, colorlinks=false} 

\usepackage{imakeidx}
\usepackage[utf8]{inputenc}
\usepackage{lmodern}
\usepackage[mathstyleoff]{breqn}
\usepackage{mathdots}
\usepackage{mathtools}
\usepackage{microtype}
\usepackage{multirow}

% add authoryear below for presentation citation
\usepackage[square, numbers, comma, sort&compress]{natbib}  % Use the "Natbib" style for refs
\usepackage{pgffor}  % For repeating patterns
\usepackage{pdflscape}
\usepackage{pgfplots}
    \pgfplotsset{compat=1.16}
\usepackage{siunitx}
\usepackage{slashed}
\usepackage{stmaryrd}
\usepackage{tabu}
\usepackage{tabularx}
\usepackage[most]{tcolorbox}
\usepackage{tensor}
\usepackage[normalem]{ulem}
\usepackage{verbatim}  % Needed for the "comment" environment to make LaTeX comments
\usepackage{Lib/vector}  % Allows "\bvec{}" and "\buvec{}" for "blackboard" style bold vectors
\usepackage{wrapfig} 
\usepackage{xcolor}
\usepackage{xfrac}
\usepackage{xspace}
\usepackage[all]{xy}


% --------------------------------------------------------------------------------------------
% -------------------------------------------TIKZ---------------------------------------------
% --------------------------------------------------------------------------------------------
% sends warning etex btwb
\usepackage{tikz}
\usepackage{tikz-cd}
\usetikzlibrary{decorations.markings}
\usepackage{tkz-euclide}
% remember that luluatex is needed for automatic placement BUT i have disabled the warning in
% /usr/local/texlive/2020/texmf-dist/tex/latex/tikz-feynman/tikzlibraryfeynman.code.tex 
\usepackage{tikz-feynman} 
\usepackage[tikz]{bclogo}                   % For cute logo boxes

%%%%%%%%%%%%%%%%%%%%%%%%%%%%%%%%%%%%%%%%%%%%
%%% TIKZ - for drawing Feynman diagrams %%%%
%%% ... use with pdflatex               %%%%
%%%%%%%%%%%%%%%%%%%%%%%%%%%%%%%%%%%%%%%%%%%%

\usetikzlibrary{arrows,shapes}
\usetikzlibrary{trees}
\usetikzlibrary{decorations}
\usetikzlibrary{matrix,arrows}              % For commutative diagram
                                            % http://www.felixl.de/commu.pdf
\usetikzlibrary{positioning}                % For "above of=" commands
\usetikzlibrary{calc,through}               % For coordinates
\usetikzlibrary{decorations.pathreplacing}  % For curly braces

\usetikzlibrary{decorations.pathmorphing}   % For Feynman Diagrams
\usetikzlibrary{decorations.markings}
\usetikzlibrary{intersections}              % For interesections


% Based on:
% http://www.texample.net/tikz/examples/feynman-diagram/

\tikzset{
    % >=stealth', %% more traditional arrows, I don't like them
    vector/.style={decorate, decoration={snake}, draw},
    fermion/.style={postaction={decorate},
        decoration={markings,mark=at position .55 with {\arrow{>}}}},
    fermionbar/.style={draw, postaction={decorate},
        decoration={markings,mark=at position .55 with {\arrow{<}}}},
    fermionnoarrow/.style={},
    gluon/.style={decorate,
        decoration={coil,amplitude=4pt, segment length=5pt}},
    % scalar/.style={dashed, postaction={decorate},
    %     decoration={markings,mark=at position .55 with {\arrow{>}}}},
    scalar/.style={dashed},
    scalarbar/.style={dashed, postaction={decorate},
        decoration={markings,mark=at position .55 with {\arrow{<}}}},
    scalarnoarrow/.style={dashed,draw},
%
%%%     Special vectors (when you need to fine-tune wiggles)
%   provector/.style={decorate, decoration={snake,amplitude=2.5pt}, draw},
%   antivector/.style={decorate, decoration={snake,amplitude=-2.5pt}, draw},
%       electron/.style={draw=black, postaction={decorate},
%        decoration={markings,mark=at position .55 with {\arrow[draw=black]{>}}}},
%   bigvector/.style={decorate, decoration={snake,amplitude=4pt}, draw},
    vectorscalar/.style={loosely dotted,draw=black, postaction={decorate}},
}

% -------------------------------------------------------------------------------------------
% ---------------------------------------ENVIRONMENT---------------------------------------
% \newenvironment{claim}[1]{\par\noindent\underline{Claim:}\space#1}{}
% \newenvironment{claimproof}[1]{\par\noindent\underline{Proof:}\space#1}{\hfill $\blacksquare$}

% -----------------------------------------COMMANDS-----------------------------------------

% \newcommand\numberthis{\addtocounter{equation}{1}\tag{\theequation}}
% \newcommand\sslashed[1]{{\mathllap{\slashed{#1}}}} replaced by \not \psi .... much better 
\renewcommand\CancelColor{\color{red}}

\def\mathcolor#1#{\@mathcolor{#1}}
\def\@mathcolor#1#2#3{%
  \protect\leavevmode
  \begingroup
    \color#1{#2}#3%
  \endgroup
}


% --------------------------------------------------------------------------------------------

% Theorems
\theoremstyle{definition}
\newtheorem*{aim}{Aim}
\newtheorem*{assumption}{Assumption}
\newtheorem*{axiom}{Axiom}
\newtheorem*{claim}{Claim}
\newtheorem{corollary}{Corollary}[subsection]
\newtheorem*{conjecture}{Conjecture}
\newtheorem{definition}{Definition}[subsection]
\newtheorem{example}{Example}[subsection]
\newtheorem*{ex}{Exercise}
\newtheorem{fact}{Fact}[subsection]
\newtheorem*{law}{Law}
\newtheorem{lemma}{Lemma}[subsection]
\newtheorem*{notation}{Notation}
\newtheorem{note}{Note}
\newtheorem{proposition}{Proposition}[subsection]
\newtheorem*{question}{Question}
\newtheorem*{rrule}{Rule}
\newtheorem{remark}{Remark}[subsection]
\newtheorem{theorem}{Theorem}[subsection]

\newtheorem*{warning}{Warning}
\newtheorem*{exercise}{Exercise}

\newtheorem{nthm}{Theorem}[section]
\newtheorem{nlemma}[nthm]{Lemma}
\newtheorem{nprop}[nthm]{Proposition}
\newtheorem{ncor}[nthm]{Corollary}


%%%%%%%%%%%%%%%%%%%%%%%%%
%%%%% Maths Symbols %%%%%
%%%%%%%%%%%%%%%%%%%%%%%%%

% Commented out when in autocompletions
% ------------------------------------------------------------------------------------------
% ---------------------------------------Matrix groups--------------------------------------

%  don't really use below
% \newcommand{\PGL}{\mathrm{PGL}}
% \newcommand{\PSL}{\mathrm{PSL}}
% \newcommand{\PSO}{\mathrm{PSO}}
% \newcommand{\PSU}{\mathrm{PSU}}
% \newcommand{\Spin}{\mathrm{Spin}}
% \newcommand{\Sp}{\mathrm{Sp}}
% ------------------------------------------------------------------------------------------
%------------------------------------- Matrix algebras -------------------------------------
% \newcommand{\gl}{\mathfrak{gl}}
% \newcommand{\ort}{\mathfrak{o}}
% \newcommand{\so}{\mathfrak{so}}
% \newcommand{\su}{\mathfrak{su}}
% \renewcommand{\sl}{\mathfrak{sl}}
% \newcommand{\uu}{\mathfrak{u}}

% ------------------------------------------------------------------------------------------
% --------------------------------------- Special sets ---------------------------------------
% \newcommand{\Cx}{\mathbb{C}}
% \newcommand{\CP}{\mathbb{CP}}
% \newcommand{\RP}{\mathbb{RP}}

% conflits with greek letters
% \newcommand{\N}{\mathbb{N}}
% \newcommand{\Q}{\mathbb{Q}}
% \newcommand{\R}{\mathbb{R}}
% \newcommand{\Z}{\mathbb{Z}}
% % don't use those
% \newcommand{\GG}{\mathbb{G}}
% \renewcommand{\H}{\mathbb{H}}
% \newcommand{\T}{\mathbb{T}}

% -------------------------------------------------------------------------------------------
% ----------------------------------------- Brackets -----------------------------------------
% \newcommand{\abs}[1]{\left\lvert #1\right\rvert}
% \newcommand{\bket}[1]{\left\lvert #1\right\rangle}
% \newcommand{\brak}[1]{\left\langle #1 \right\rvert}
% \newcommand{\braket}[2]{\left\langle #1\middle\vert #2 \right\rangle}
% \newcommand{\bra}{\langle}
% \newcommand{\ket}{\rangle}
% \newcommand{\norm}[1]{\left\lVert #1\right\rVert}
% \newcommand{\normalorder}[1]{\mathop{:}\nolimits\!#1\!\mathop{:}\nolimits}
% \newcommand{\tv}[1]{|#1|}
% \renewcommand{\vec}[1]{\boldsymbol{\mathbf{#1}}}

% ----------------------------------------- not-math -----------------------------------------
% \newcommand{\bolds}[1]{{\bfseries #1}}
% \newcommand{\cat}[1]{\mathsf{#1}}
% \newcommand{\ph}{\,\cdot\,}
% \newcommand{\term}[1]{\emph{#1}\index{#1}}
% \newcommand{\phantomeq}{\hphantom{{}={}}}

% -----------------------------------------------------------------------------------------
%----------------------------------------- Algebra-----------------------------------------
% \DeclareMathOperator{\adj}{adj}
% \DeclareMathOperator{\Aut}{Aut}
% \DeclareMathOperator{\End}{End}
% \DeclareMathOperator{\Hom}{Hom}
% \DeclareMathOperator{\id}{id}
% \DeclareMathOperator{\im}{im}
% \DeclareMathOperator{\tr}{tr}
% \DeclareMathOperator{\Tr}{Tr}
% \DeclareMathOperator{\Span}{\mathrm{Span}}
% \newcommand{\Der}{\mathrm{Der}}
% \newcommand{\Cech}[1]{\check{#1}}

% don't use those
% \newcommand{\Gr}{\mathrm{Gr}}
% \DeclareMathOperator{\Ann}{Ann}
% \DeclareMathOperator{\Char}{char}
% \DeclareMathOperator{\disc}{disc}
% \DeclareMathOperator{\dom}{dom}
% \DeclareMathOperator{\fix}{fix}
% \DeclareMathOperator{\image}{image}
% \newcommand{\Bilin}{\mathrm{Bilin}}
% \newcommand{\Frob}{\mathrm{Frob}}

% -------------------------------------------------------------------------------------------
% ---------------------------------------- Topology  ----------------------------------------
% \newcommand{\pr}{\mathrm{pr}}
% \newcommand{\pt}{\mathrm{pt}}
% \newcommand{\dR}{\mathrm{dR}}
% \newcommand{\Hol}{\mathrm{Hol}}
% \newcommand{\hol}{\mathfrak{hol}}


% ------------------------------------------ Others ----------------------------------------- 

% % should autocomplete this
% \renewcommand{\d}{\mathrm{d}}

% \newcommand\Art{\mathrm{Art}} %typo here?
% \newcommand{\B}{\mathcal{B}}
% \newcommand{\cU}{\mathcal{U}}
% \newcommand{\D}{\mathrm{D}}
% \newcommand{\exterior}{\mathchoice{{\textstyle\bigwedge}}{{\bigwedge}}{{\textstyle\wedge}}{{\scriptstyle\wedge}}}
% \newcommand{\F}{\mathbb{F}}
% % \newcommand{\G}{\mathcal{G}}
% \newcommand{\haut}{\mathrm{ht}}
% \newcommand{\Id}{\mathrm{Id}}
% \newcommand{\lie}[1]{\mathfrak{#1}}
% \newcommand{\op}{\mathrm{op}}
% \newcommand{\Oc}{\mathcal{O}}
% \newcommand{\Ps}{\mathcal{P}}
% \newcommand{\qeq}{\mathrel{``{=}"}}
% \newcommand{\Rs}{\mathcal{R}}
% \newcommand{\Vect}{\mathrm{Vect}}
% \newcommand{\wsto}{\stackrel{\mathrm{w}^*}{\to}}
% \newcommand{\wt}{\mathrm{wt}}
% \newcommand{\wto}{\stackrel{\mathrm{w}}{\to}}
% \renewcommand{\P}{\mathbb{P}}
% %\renewcommand{\F}{\mathcal{F}}


% \let\Im\relax
% \let\Re\relax

% % should autocomplete those
% \DeclareMathOperator{\Diff}{Diff}
% \DeclareMathOperator{\Ext}{Ext}
% \DeclareMathOperator{\Sym}{Sym}

% \DeclareMathOperator{\area}{area}
% \DeclareMathOperator{\card}{card}
% \DeclareMathOperator{\ccl}{ccl}
% \DeclareMathOperator{\ch}{ch}
% \DeclareMathOperator{\cl}{cl}
% \DeclareMathOperator{\cls}{\overline{\mathrm{span}}}
% \DeclareMathOperator{\coker}{coker}
% \DeclareMathOperator{\conv}{conv}
% \DeclareMathOperator{\cosec}{cosec}
% \DeclareMathOperator{\cosech}{cosech}
% \DeclareMathOperator{\covol}{covol}
% \DeclareMathOperator{\diag}{diag}
% \DeclareMathOperator{\diam}{diam}
% \DeclareMathOperator{\energy}{energy}
% \DeclareMathOperator{\erfc}{erfc}
% \DeclareMathOperator{\erf}{erf}
% \DeclareMathOperator*{\esssup}{ess\,sup}
% \DeclareMathOperator{\ev}{ev}
% \DeclareMathOperator{\fst}{fst}
% \DeclareMathOperator{\Fit}{Fit}
% \DeclareMathOperator{\Frac}{Frac}
% \DeclareMathOperator{\Gal}{Gal}
% \DeclareMathOperator{\gr}{gr}
% \DeclareMathOperator{\hcf}{hcf}
% \DeclareMathOperator{\Im}{Im}
% \DeclareMathOperator{\Ind}{Ind}
% \DeclareMathOperator{\Int}{Int}
% \DeclareMathOperator{\Isom}{Isom}
% \DeclareMathOperator{\lcm}{lcm}
% \DeclareMathOperator{\length}{length}
% \DeclareMathOperator{\Lie}{Lie}
% \DeclareMathOperator{\like}{like}
% \DeclareMathOperator{\Lk}{Lk}
% \DeclareMathOperator{\Maps}{Maps}
% \DeclareMathOperator{\orb}{orb}
% \DeclareMathOperator{\ord}{ord}
% \DeclareMathOperator{\otp}{otp}
% \DeclareMathOperator{\poly}{poly}
% \DeclareMathOperator{\rank}{rank}
% \DeclareMathOperator{\rel}{rel}
% \DeclareMathOperator{\Rad}{Rad}
% \DeclareMathOperator{\Re}{Re}
% \DeclareMathOperator*{\res}{res}
% \DeclareMathOperator{\Res}{Res}
% \DeclareMathOperator{\Ric}{Ric}
% \DeclareMathOperator{\rk}{rk}
% \DeclareMathOperator{\Rees}{Rees}
% \DeclareMathOperator{\Root}{Root}
% \DeclareMathOperator{\sech}{sech}
% \DeclareMathOperator{\sgn}{sgn}
% \DeclareMathOperator{\snd}{snd}
% \DeclareMathOperator{\Spec}{Spec}
% \DeclareMathOperator{\spn}{span}
% \DeclareMathOperator{\stab}{stab}
% \DeclareMathOperator{\St}{St}
% \DeclareMathOperator{\supp}{supp}
% \DeclareMathOperator{\Syl}{Syl}
% \DeclareMathOperator{\vol}{vol}

% % who cares?
% % --------------------------------------- Probability ---------------------------------------
% \DeclareMathOperator{\Bernoulli}{Bernoulli}
% \DeclareMathOperator{\betaD}{beta}
% \DeclareMathOperator{\bias}{bias}
% \DeclareMathOperator{\binomial}{binomial}
% \DeclareMathOperator{\corr}{corr}
% \DeclareMathOperator{\cov}{cov}
% \DeclareMathOperator{\gammaD}{gamma}
% \DeclareMathOperator{\mse}{mse}
% \DeclareMathOperator{\multinomial}{multinomial}
% \DeclareMathOperator{\Poisson}{Poisson}
% \DeclareMathOperator{\var}{var}
% \newcommand{\E}{\mathbb{E}}
% \newcommand{\Prob}{\mathbb{P}}