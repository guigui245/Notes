% !TeX root = ../main.tex
% 
\newpage
\lhead{\emph{Lecture 2}}
\section{Lie Groupoids} % (fold)
\label{sec:lie_groupoids}
\subsection{Definition and structure maps} % (fold)
\label{sub:definition_and_structure_maps}
Some notation will use some category theory, so we recall here that a small category is a category where objects and morphisms are \textit{small}. This means that they they are small enough to fit in the category of \textbf{Set}.
\definition{
    A Groupoid $\mathcal{G} \coloneq (G_1 \rightrightarrows G_0)$ is a \textit{small category} with morphisms that are all invertible.
    Let's unpack this definition (or parts of it for now):
    \begin{itemize}
        \item $G_0$ is the set of objects
        \item $G_1$ is the set of morphisms
        \item An element $g \in G_1$ is denoted, given a pair of functions called the \textit{source} and \textit{target} $s,t: G_1 \rightrightarrows G_0$ such that $s(g) \rightarrow t(g)$. 
        Technically, this defines the set of composable arrows $G_1 ~_s{\times}_t \:G_1$.
    \end{itemize}
}
% subsection definition_and_structure_maps (end)
\subsection{Bisections} % (fold)
\label{sub:bisections}

% subsection bisections (end)
\subsection{Lie Groupoids} % (fold)
\label{sub:lie_groupoids}

% subsection lie_groupoids (end)
\subsection{Examples of Lie Groupoids} % (fold)
\label{sub:examples_of_lie_groupoids}

% subsection examples_of_lie_groupoids (end)

\subsection{Morphisms of Lie Groupoids} % (fold)
\label{sub:morphisms_of_lie_groupoids}

% subsection morphisms_of_lie_groupoids (end)
\subsection{Vector fields on Lie Groupoids} % (fold)
\label{sub:vector_fields_on_lie_groupoids}

% subsection vector_fields_on_lie_groupoids (end)
\subsection{Action Lie Groupoid} % (fold)
\label{sub:action_lie_groupoid}
We follow \citep{Jurco2019a} for a brief but useful description of the action of Lie groupoids on manifolds.
We recall that the orbit space  $M / G$ of a group action $\rho: G \times M \rightarrow M$ by a Lie group $G$ can be badly behaved (eg: not free, so singular quotient space). One way out of this problem is to consider the action of a Lie groupoid. \\

\definition{
    The action Lie groupoid of a group $G$ on manifold $M$ is the category $G \ltimes M \rightrightarrows M$ (sometimes denote $M // G$) or $M//_\rho G$ for map $\rho$ an automorphism on the manifold).
    For $g \in G$ and $x \in M$, the map
    \begin{equation*}
        x \mapsto g \triangleright x
    \end{equation*}
    There is a good review about this topic on \href{https://unapologetic.wordpress.com/2007/06/09/groupoids-and-more-group-actions/}{blog entry}.
    Talk about functor $M//G \rightrightarrows G$ that is faithful.
}


% subsection action_lie_groupoid (end)
% section lie_groupoids (end)