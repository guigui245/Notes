% !TeX root = ../main.tex
% 
\newpage
\lhead{\emph{Lecture 3}}
\section{Lie Algebroids} % (fold)
\label{sec:lie_algebroids}
A good reference to add to the mix is \citep{Wright2019}.
\subsection{Vector Bundles} % (fold)
\label{sub:vector_bundles}
Let's first review facts about vector bundles, but with a more `categorical' mindset.

\definition{
    A \textbf{vector bundle} $\pi : E \rightarrow M$ over a smooth manifold $M$ is a fibre bundle whose fibre $E_x$ is a vector space $V \in \mathbf{Vect} \: \forall x \in M$. 
    The dimension of the typical fibre $E_M$ is called the rank and $\mathrm{dim}(E_M) := \mathrm{dim}(E_x) $ for all $x$.
}\label{def:vec_bundle}
% 
\definition{
    A \textit{local trivialisation} is a map $\varphi$ such that on $U \subset M$ open, $\varphi : \pi^{-1}(U) \rightarrow U \times V$ is a diffeomorphism.\\
    On overlaps $U_1 \cap U_2 \subset M$, local trivialisations define $\mathrm{GL}(V)$-valued \textit{transition functions}.
}\label{def:local trivialisation}
% 
\definition{
    A smooth map $F$ between $2$ vector bundles $F: E_1 \rightarrow E_2$ is a bundle \textbf{vector bundle morphism} if there exists smooth map $\phi \in C^{\infty}(M_1, M_2)$ between the bases such that 
    \begin{equation}
        \begin{tikzcd}
            E_1 \arrow[r, "F"] \arrow[d, "\pi_1"] & E_2 \arrow[d, "\pi_2"] \\
            M_1 \arrow[r, "\phi"] & M_2
        \end{tikzcd}\label{diag:vbundle_morphism}
    \end{equation}
    commutes. We say that $F$ is a \textit{covering} for $\phi$.
    Equivalently, $F$ restricts to a linear map on the fibre
    $F_x: (E_1)_x \rightarrow (E_2)_{\phi(x)}$.
}

\definition{
    Vector bundles over smooth manifolds with vector bundle morphism forms the \textbf{category of vector bundles} denoted $\mathbf{Vect}_{\mathrm{Man}}$.
}\label{def:vect_man_category}

\remark[Categorification]{
    Fixing a base manifold $M$, the point-wise construction of fibre bundles over $M$ restricts us to the subcategory of vector bundles over $M$ denoted $\mathbf{Vect}_M$.
    % This subcategory has objects the vector bundles over $M$ and morphisms $F$ covering the identity morphism.
    % Further, point-wise constructions are symmetric abelian products $ \bigotimes : \mathbf{Vect}_M \times \mathbf{Vect}_M \rightarrow \mathbf{Vect}_M$ such that $\bigotimes$ is a \textbf{binary functor} (or bifunctor).
    % A bifunctor for morphisms $F,G,H \in \mathrm{Mor}(\mathbf{Vect}_M)$, has
    % \begin{equation}
    % (F \circ G )\otimes H = (F \otimes H) \circ (G \otimes H)
    % \end{equation}
    % and so on.
    % A symmetric product in this category means there exists a symmetric map
    % \begin{equation}
    %     S : \mathbf{Vect}_M \times \mathbf{Vect}_M \rightarrow \mathbf{Vect}_M \times \mathbf{Vect}_M.
    % \end{equation}
    % And abelianness, I'm not sure
    % \footnote{
    %     expand on this please
    % }
    So the point-wise construction of vector bundle over base manifold $M$ forms a \textbf{abelian symmetric monoidal category}. 
    \footnote{
        One day, try to understand category stuff
    }
}\label{rem:categorical_aspects_of_vbundle}

\definition{
    Given vector bundle $E_1, E_2$ over manifold $M$, 
    \begin{itemize}
        \item the vector bundle direct sum called the \textit{Whitney sum} is the fiberwise direct sum $E_1 \boxplus E_2$ (as seen in \cref{def:product_poisson_mfd}),
        \item the vector bundle tensor product is the fiberwise tensor product $(E_1)_x \otimes (E_2)_x$ for all $x \in M$,
        \item and the dual vector bundle is given by $E^* = \mathrm{Hom}(E, \mathbb{R})$ (which is a functor by the way).
    \end{itemize}
}\label{def:vb_sum_prod_dual}

\definition{
    Given a vector bundle $E \xrightarrow{\pi} M$, and a smooth map $\phi:N \rightarrow M$, \textbf{categorical pulback} of $E$ by $\phi$ is defined as $\phi^* E = N  \tensor[_\phi]{\times}{_\pi} E$ such that the diagram commutes:
    \begin{equation}
        \begin{tikzcd}
            N \tensor[_\phi]{\times}{_\pi} E \arrow[r, ""] \arrow[d, ""]& E  \arrow[d, "\pi"] \\
            N \arrow[r, "\phi"] &M 
        \end{tikzcd}
    \end{equation}
    which is simply $\phi^*E = \{ (p, e) \in N \times E \; | \; \phi(p) = \pi(e)\}$.
    We will see in more detail what it entails shortly.
}\label{def:categorical pullback of vector bundle}
% 
\definition{
    A \textbf{section} $s$ of vector bundle $E \xrightarrow{\pi} M$ is a map $s : M \rightarrow E$ such that $\pi \circ s = \mathrm{id}_M$. 
    The set of all sections of $E$ forms a $C^\infty(M)$-module and is denoted
    \begin{equation}
        \Gamma(E) = \{s : M \rightarrow E\; |\;\pi \circ s = \mathrm{id}_M\}.
    \end{equation}
}\label{def:sections_vb}

\definition{
    On an open set $U \subset M$, a basis of sections (or \textbf{frame}) is a set 
    \begin{equation}
    \{e_i : U \rightarrow \pi^{-1}(U) \vert \pi \circ e_i = \mathrm{id}_U\}_{i=1}^{\mathrm{rk}(E)}    
    \end{equation}
    that defines a local trivialisation (\cref{def:local trivialisation}). 
    If such sections are globally defined, the bundle is trivial (or trivialisable).
}\label{def:frame of vector bundle}
% 
\remark{
    We would naturally guess that the functor $C^\infty : \mathbf{Man} \rightarrow \mathbf{Ring}$ would similarly extend to sections on manifold. 
    But the assignment $\Gamma : \mathbf{Vect}_{\mathrm{Man}} \rightarrow R\textrm{-}\mathbf{Mod}$ fails to be a functor. 
    To see this, consider the vector bundle morphism $F: E_1 \rightarrow E_2$ covering $\phi: M \rightarrow N$, and the pullback bundle along $\phi$:
    % 
    \begin{equation}
        \begin{tikzcd}[sep = huge]
            E_1 \arrow[r, "F"] \arrow[d, "\pi_1"] 
            &\phi^*E_2 \arrow[r, "\mathrm{id}_{E_2}"] \arrow[d, "\mathrm{pr}_2", swap]
            & E_2  \arrow[d, "\pi_2", swap] \\
            % 
            M \arrow[r, "\mathrm{id}_M"] \arrow[bend left= 30, "s_1", dashed]{u}
            & M \arrow[r, "\phi"] \arrow[bend right = 40, swap, color = red, "\phi^*s_2"]{u}
            & N  \arrow[u, bend right = 30, "s_2", dashed, swap]
        \end{tikzcd}
    \end{equation}
    For sections on the vector bundle, we have the maps
    \begin{align}
        &\Gamma(E_1) \xrightarrow{F} \Gamma(\phi^*E_2) \xleftarrow{\phi^*} \Gamma(E_2)\\
        &s_1 \rightarrow F \circ s_1 \;;\; s_2 \circ \phi \leftarrow s_2. \nonumber
    \end{align}
    Meaning that regardless of whether $\phi$ is diffeomorphic, we can push sections $s_1 \in \Gamma(E_1)$ to sections $F(s_1) \in \phi^*E_2$ and sections $s_2 \in \Gamma(E_2)$ can be pullbacked to $\phi^* s_2 \in \Gamma(\phi^* E_2)$.
    \\If $\phi$ is \textbf{not} a diffeomorphism \footnotemark, then one \textit{cannot} in general construct such maps.
    However, when maps agree
    \begin{equation}
        F \circ s_1 = s_2 \circ \phi \qquad \Leftrightarrow \qquad s_2 \sim_F s_1 
    \end{equation}\label{eqn:F_related}
    we say that the sections are \textbf{F-related}.
}\label{rem:duality_not_functor}
\footnotetext{
    which it isn't for sigma models.
}
% 
\definition{
    If $\phi :M \rightarrow N$ \textbf{is} a diffeomorphism, and $F:E_1 \rightarrow E_2$ a covering of vector bundle then the \textbf{pushforward} of sections
    \begin{align}
        F_*: \Gamma(E_1) &\rightarrow \Gamma(E_2) \\
        s_1 &\mapsto F \circ s_1 \circ \phi^{-1}
    \end{align}
    is well-defined.
    The pushforward satisfies
    \begin{align}
        F_*(s+r) &= F_*(s) + F_*(r)\\
        F_*(f \cdot s) &= (\phi^{-1})^*f \cdot F_*(s)
    \end{align}
    for $s,r \in \Gamma(E_1)$ and $f \in C^\infty(M)$. 
}\label{def:pushforward_sections}
% 
\remark{
    Note that you can \textit{always} pushforward tangent vectors via the differential map $\phi_*: TM \rightarrow TN$, but to pushforward vector fields, you need $\varphi$ to be a diffeomorphism.
}
% 
\definition{
    In contrast, $\phi$ need not be a diffeomorphism for the \textbf{pullback} of dual sections (aka: forms) to be well-defined.
    \begin{align*}
        F: \Gamma(E_2^*) &\rightarrow \Gamma(E_1^*)\\
        \alpha &\mapsto F^* \alpha = \alpha \circ \phi
    \end{align*}
    So for any sections $s_1 \in \Gamma(E_1)$, 
    \begin{equation}
        F^*\alpha (s_1) = \left(\alpha \circ \phi\right) ( F \circ s_1)
    \end{equation}
    since $F(s_1) \in \Gamma( \phi^* E_2)$. 
    % Meaning that F-relatedness is not necessary to construct well-defined forms since we need only pushforward
    % We see that that the notion of pullback and pushforward allows us to extend to notion of F-relatedness as in \eqref{eqn:F_related}
    \footnote{
    In the case of cotangeant bundle, I should expand on the fact that $\phi$ needs to be diffeomorphism to define pullbacks. See notes on int systems    
    }
}\label{def:pushforward}
% 
\definition{
    The pullback naturally extends to all tensor powers of the dual bundles $\bigotimes^{k}E_2^*$, such that for $\eta, \omega \in \Gamma( \bigotimes^\bullet E_2^*)$, and function $f \in \Gamma(\bigotimes^0 E_2) \cong C^\infty(N)$, we have
    \begin{align*}
        F^*(\eta + \omega) &= F^* \eta + F^* \omega \\
        F^*(\eta \otimes \omega ) &= F^*\eta \otimes F^*\omega \\
        F^*(f) &= f \circ \phi = \phi^* f.
    \end{align*}
}\label{def:extensions_to_tensor_duals}
% 
Importantly, we can pullback vector bundle through smooth maps by extending the map $\phi^*: \Gamma(E) \rightarrow \Gamma(\phi^* E)$ over the ring of functions of the target $\mathcal{C}^\infty(N)$.
\proposition{
    Let $\phi: M \rightarrow N$ a smooth map between manifolds and let $E \rightarrow N$ a vector bundle over $N$.
    Then the \textbf{pullback vector bundle} is isomorphic to
    \begin{equation}
        \Gamma( \phi^*E) = \mathcal{C}^\infty(M) \otimes_{\mathcal{C}^\infty(N)} \Gamma(E),
    \end{equation}
    Apparently this is an example of the Serre-Swan theorem but not today.
}\label{prop:Pullback bundle isomorphism}
% 

\definition{
    Given vector bundle morphisms $F: E_1 \rightarrow E_2$, $G: E_2 \rightarrow E_3$, we see that on the tensor powers of dual bundles we have the contravariant identity:
    \begin{equation}
        (G \circ F)^* = F^* \circ G^*.
    \end{equation}
    This motivates the definition of the \textit{contravariant} \textbf{tensor functor} to the category of associative unital algebras
    \begin{align*}
        \mathcal{T}: \mathbf{Vect}_{Man} &\rightarrow \mathbf{AssAlg}\\
        E &\mapsto \Gamma\left( \bigotimes {}^\bullet E^*\right) \\
        F &\mapsto F^*
    \end{align*}
}\label{def:tensor_functor}
Look into spanning functions and restrictions of $C^\infty$ functor.

\definition{
    Let $E \rightarrow M$ be a smooth vector bundle, a \textbf{Koszul connection} on the sections of $E$ is an $\mathbb{R}$-linear map 
    \begin{align*}
        \nabla: \mathfrak{X}(M) \times \Gamma(E) &\rightarrow \Gamma(E)\\
        (X,s) & \mapsto \nabla_X s
    \end{align*}
    such that the Leibniz identity is satisfied.
    \begin{equation}
        \nabla(f \cdot s) = f \nabla s + df \cdot s
    \end{equation}
}\label{def:kozul connection}
% 

\definition{
    Given a vector bundle $E \rightarrow M$, we construct the \textbf{Koszul complex} on sections of the vector bundle.
    We define a boundary map $\delta : \Gamma(\bigwedge^q E) \rightarrow \Gamma(\bigwedge^{q-1} E)$ such that
    \begin{equation}
        \delta (e_1 \wedge \ldots \wedge e_q) = \sum_{i=1}^{q}(-)^{i+1} \rho(e_i) (e_1 \wedge \ldots \hat{e_i} \wedge \ldots \wedge e_q)
    \end{equation}
    for $e_i \in \Gamma(E)$ and $\rho \in \Gamma(E^*)$. This map enjoys the property $\delta^2=0$, from which we can take the \textit{homology} of this complex.
}

\definition{
    Let $E \rightarrow M$ be a vector bundle, the space of  \textbf{$E$-valued $p$-forms} is denoted
    \begin{equation}
        \Omega^p(M;E) = \Gamma(E \otimes \bigwedge^p T^*M).
    \end{equation}
    Given a Koszul connection $\nabla$ on $E$, there is a unique way to extend $\nabla$ to an \textbf{exterior covariant derivative} $d^\nabla$, mapping
    \begin{equation}
        d^\nabla : \Omega^p(M;E) \rightarrow \Omega^{p+1}(M;E).
    \end{equation}
    If the connection $d^\nabla$ is flat, then this forms the \textit{Koszul complex} with $(d^\nabla)^2 = 0$.\\
    We extend linearly from 
    \begin{align*}
        d^\nabla( \omega \otimes s) &= d \omega \otimes s + (-)^{|\omega|} \omega \otimes d^\nabla s \\
        (d^\nabla s) X &= \iota_X d s 
    \end{align*}
    for $s \in \Gamma(E), \omega \in \Omega^p(M;E), X \in TM$.
}\label{def:exterior covariant derivative}
% subsection vector_bundles (end)
% 
\subsection{Definition and Examples} % (fold)
\label{sub:definition_and_examples}

\definition{
    A \textbf{Derivative Lie algebra} is a vector bundle $\alpha: A \rightarrow M$ whose $\mathbb{R}$-linear module of sections $(\Gamma(A), [\cdot,\cdot])$ acts as a derivation on each of its arguments.
    That is 
    \begin{equation*}
         \mathrm{ad}_{[\cdot,\cdot]} : \Gamma(A) \rightarrow \mathrm{Der}(A)
    \end{equation*} 
    is well-defined, which in this case means that the derivation property respects the $C^\infty(M)$-module structure.
}\label{def:derivative_lie_algebra}
\example{
    A Poisson algebra is a derivative Lie algebra.
}
% 
\definition{
    A \textbf{Lie Algebroid} $\left\{ A \rightarrow M; \rho: A \rightarrow TM; [\cdot, \cdot] \right\}$
    over smooth manifold $M$ is a vector bundle $A \rightarrow M$ together with a derivative Lie algebra structure on $\left(\Gamma(A), \left[\cdot,\cdot\right]\right)$ and a vector bundle morphism $\rho: A \rightarrow TM$ called the \textit{anchor map}. \\
    The anchor map induces a Lie algebra homomorphism between $\mathcal{C}^\infty(M)$-modules of sections 
    \begin{equation}
        \rho_* : \Gamma(A) \rightarrow \Gamma(TM) 
    \end{equation}
    by the \textit{symbol-squiggle theorem} which we will see in \ref{thm:symbol_squiggle}
    with
    \begin{equation}
        \rho_*[s_1,s_2]_A = [\rho(s_1), \rho(s_2)]_{TM}
    \end{equation}
    Being a \textit{derivative Lie algebra} means that for $s_1,s_2, s_3 \in \Gamma(A)$ and $f \in C^{\infty}(M)$, the following equations are satified:
    \begin{align}
        \left[s_1, f \cdot s_2\right]_A &= f  \cdot [s_1,s_2]_A + \rho_*(s_1)[f] \cdot s_2
        \label{eq:lie algebroid axioms 1}
        \\
        \left[s_1, [s_2,s_3]_A\right]_A &= \left[[s_1,s_2]_A, s_3\right]_A + \left[s_2, [s_1, s_3]_A\right]_A   \label{eq:lie algebroid axioms 2}
    \end{align}
    Locally, if the algebroid structure is given by structure functions $\tensor{C}{_{ab}^c}(x) \in \Gamma( \Lambda^2 A^* \otimes A)$ in a chart $U_\alpha$ with coordinates $x^i$.
    Considering local structure functions as two-form valued sections on the algebroid, we deduce a Jacobi type identity, with the derivative map given by the pushforward $\rho_*$:
    \begin{equation}
        \tensor{C}{_{[ab}^e} \tensor{C}{_{c]e}^d} + \frac{\partial}{\partial x^i}\tensor{C}{_{[ab}^d} \tensor{\rho}{^i_{c]}} = 0,
        \label{eq:Jacobi identity for Lie algebroids}
    \end{equation}
    where we have put $\rho_\star^i(\xi_a) = \tensor{\rho}{^i_a}$ is the components of vector field associated to a local trivialisation when anchored down to the tangent bundle.
    Unlike Leibniz algebras (\cref{def:Liebniz algebra}), the bracket over sections is skew-symmetric.
}\label{def:lie_algebroids}
% 
% 
\remark{
    In general, vector bundle morphism \textbf{DO NOT} induce well-defined maps between modules of sections because $\Gamma$ fails to be a functor in if the map between spaces is not a diffeomorphic (\cref{rem:duality_not_functor}).
    We explore this further in \cref{sub:morphisms_of_la}.
}\label{rem:double pullback lie algebroid}
% 
% 
\remark{
    The anchor map naturally defines two distributions $\mathrm{ker}(\rho) \subset A$ and $\mathrm{im}(\rho) \subset TM$.
    Now fiberwise, the vector spaces $\mathfrak{g}_x := \mathrm{ker}(\rho_x) \subset A_x$ form a Lie algebra with bracket extending on the fibres $A_x$.
    We call $(\mathfrak{g}_x, [\cdot,\cdot]_{\Gamma(A)}|_x)$ the \textbf{isotropy Lie algebras} of Lie algebroid $A$.\\
    SInce $\rho$ is a Lie algebra morphism, the image distribution $\rho(A)$ is involutive and it will be integrable by a singular foliation on $M$.
    We call the image distribution the \textbf{characteristic distribution} of the Lie algebroid $A$.
} 
% 
\example{
% 
    \begin{itemize}
        \item A natural first example is the vector bundle $\mathfrak{g} \rightarrow \star$, with $\mathfrak{g}$ having a Lie algebra structure. This is a trivial Lie algebroid.
        \item The tangent bundle $TM \rightarrow M$ is a Lie algebroid with projection anchor map and the vector field forming a derivative algebra.
        \item Likewise, involutive distributions $D \hookrightarrow TM$ with trivial anchor form a sub Lie algebroid.
        \item The vector bundle $\mathbb{R} \rightarrow M$, with derivative Lie algebra being the commutative algebra of smooth functions over $M$.
        Then for $h \in C^\infty(M)$ and associated vector field $X_h \in \Gamma(TM)$, the anchor map is given by
        \begin{equation*}
            \rho_X : f \mapsto f \cdot X
        \end{equation*}
        and the bracket is
        \begin{equation}
            [f,g]_X = f X(g) - g X(f).
        \end{equation}
        We've just reinterpreted the regular Lie algebra of functions over a manifold as a Lie algebroid.
        \item The Atiyah algebroids $A_P$ of a principal $G$-bundle $\pi: P \rightarrow M$, as a vector bundle appears in the sequence 
        \begin{equation*}
            0 \rightarrow \mathrm{ad}(P) \rightarrow A_P \rightarrow TM \rightarrow 0,
        \end{equation*}
        where the adjoint bundle $\mathrm{ad}(P) \cong P \times_G \mathfrak{g}$ of Lie algebra $\mathfrak{g}$ is the associated bundle of $P$ on $\mathfrak{g}$ by $G$. 
        This is constructed from the short exact sequence
        \begin{equation}
             0\rightarrow VP\rightarrow TP \xrightarrow{\pi_*} TM \rightarrow 0,
        \end{equation}
        where $VP$ is the \textit{vertical bundle}, the kernel of the differential surjective map. Now the vertical bundle is isomorphic to the trivial bundle $VP \cong P \otimes \mathfrak{g}$.
        Now since $P$ is a principal bundle, G acts on this short exact sequence yielding the Atiyah sequence as $G$ acts as the adjoint map on the vertical bundle.
        \item The bundle of derivations $DE \rightarrow M$ for a vector bundle $E \rightarrow M$ is a Lie algebroid.
    \end{itemize}
    % 
}\label{ex:Examples_of_lie_algebroids}
% subsection definition_and_examples (end)
% 
\subsection{Morphisms of Lie algebroids} % (fold)
\label{sub:morphisms_of_la}

We will talk about the notion of local Lie algebras later, but in the case of Lie algebroid, the extra structure of a vector bundle constrains morphisms between these algebraic structures.
We start by considering a general vector bundle morphism $F: A \rightarrow B$, and demand that this map induces a Lie algebra morphism of the modules of sections, which is not always guaranteed.
% 
\definition{
    Let $A \rightarrow M$ and $B \rightarrow M$ be Lie algebroids over the same base. 
    A \textbf{Lie algebroid morphism over the same base} $\Phi: A \rightarrow B$ is a vector bundle morphism (\cref{sub:morphisms_of_la}) inducing a map between smooth sections $\Gamma(B) \rightarrow \Gamma(A)$ such that the bracket is preserved
    \begin{equation}
        \Phi \circ [a_1, a_2]_A = [\Phi \circ a_1, \Phi \circ a_2]_B.
    \end{equation}
}\label{def:morphism of Lie algebroids over same base}
% 
However, when the base is different, morphism in the category of Lie algebroid are more complicated. 
But using the notion of \textit{relatedness}, a suitable constrution can be established.
Originally from \citep{Higgins1990} is presented here.

% subsection morphisms_of_la (end)
% 
\subsection{Algebraic structures associated with Lie algebroid} % (fold)
\label{sub:algebraic_structures_associated_with_la}
% 
Let's first discuss a notion that combines the structure of $\mathbbm{Z}$-graded Lie superalgebras and supercommutative rings. Attention not to confuse with regular supersymmetry and $\mathbbm{Z}_2$-grading for Poisson superalgebras for example.
\definition{
    A \textbf{Gerstenhaber algebra} $A^\bullet$ is a graded commutative algebra with a Lie bracket $\llbracket\cdot,\cdot\rrbracket$ of degree $-1$ that satisfies the \textit{Poisson identity} and a multiplication $\cdot$ in the supercommutative associative ring. For $a,b,c \in A^\bullet$, we denote $|a|$ the degree of $a$ and we have the following
    \begin{tcolorbox}
        \begin{itemize}
            \item $|ab| = |a| + |b|$
            \item $|\llbracket a,b\rrbracket| = |a| + |b| - 1 $
            \item $ab = (-1)^{|a||b|} ba$
            \item $\llbracket a, bc\rrbracket = \llbracket a,b\rrbracket c + (-1)^{(|a|-1)|b|}b \llbracket a,c\rrbracket$
            \item $\llbracket a,b\rrbracket = - (-1)^{(|a|-1)(|b|-1)} \llbracket b,a\rrbracket$
            \item $\llbracket a, \llbracket b, c\rrbracket\rrbracket = 
            \llbracket \llbracket a,b\rrbracket,c\rrbracket
            + (-1)^{(|a|-1)(|b|-1)} \llbracket b, \llbracket a,c\rrbracket\rrbracket$  
        \end{itemize}
    \end{tcolorbox}    
}\label{def:Gerstenhaber}
% 
Given a Lie algebroid $(A, \rho, [,])$ over $M$, we can \textit{uniquely} extend the bracket to a Gerstenhaber bracket on the graded algebra of multisections.
% 
\definition{
    There exists a Gerstenhaber algebra $(\Gamma(\bigwedge^{\bullet}A), \wedge, \llbracket \cdot,\cdot\rrbracket)$ such that, for $a,b \in \Gamma(A)$ and $f,g \in C^{\infty}(M)$
    \begin{itemize}
        \item $\llbracket a,b\rrbracket = [a,b]$
        \item $\llbracket a,f\rrbracket = \rho_*a [f]$
        \item $\llbracket f,g\rrbracket=0$
    \end{itemize}
}\label{def:gerstenhaber_for_Lie_algebroid}
% 
\definition{
    We can construct a Lie algebroid counterpart to exterior calculus of the differential geometry of tangent bundles. This is a dual construction to the Gerstenhaber algebra seen above.
    Given a Lie algebroid $A \rightarrow M$, we construct a differential graded algebra $\Omega(A) = \bigoplus \Omega^k(A)$ of the sections $\Gamma(\bigwedge^\bullet A^*)$.
    The map
    \begin{equation*}
         d_A: \Omega^k(A) \rightarrow \Omega^{k+1}(A)
    \end{equation*}
    is called the \textit{differential} and explicitly, acting on homogeneous element $\omega \in \Gamma(\bigwedge^k A^*)$ we have:
    \begin{equation}
        d_A \omega(a_0, \ldots, a_k) = 
        \sum_{i<j}(-1)^{i+j-1} \omega([a_i, a_j], a_1, \ldots, \hat a_i, \ldots \hat a_j \ldots, a_k)
        + \sum_{i=0}^{k}(-1)^{i-1} \rho_* a_i [\omega(a_0, \ldots \hat a_i, \ldots a_k)],
    \end{equation}
    for $a_i \in \Gamma(A)$.
    This differential graded algebra $\Omega^\bullet(A) = (\Gamma(\bigwedge^{\bullet} A^{*}), \wedge, d_A)$ is called the \textbf{exterior algebra} of the Lie algebroid $A$.
}\label{def:exterior_algebroid}

As usual we find $d_A^2 = 0$ which forms a \textit{de Rham} complex and its cohomology is the \textbf{Lie algebroid cohomology with trivial coefficients}.

% \remark{    
% Maybe a slightly more illuminating definition of the differential is given in \citep{Cattaneo2003}. For $\omega \in \Gamma(A^*)$ and $a_1,a_2 \in \Gamma(A)$, we have
% \begin{equation}
%     d_A \omega (a_1, a_2) = - \omega([a_1,a_2]) + (d_A \omega(a_1) )(a_2) - (d_A \omega(a_2))(a_1)
% \end{equation}
% This is easier to understand upon introducing the notation $\langle \cdot, \cdot \rangle$ for the natural pairing of section of the algebroid and its dual. This definition is extended to recover what we had before. And it's easier to see that this is a nilpotent derivation now.
% }
% 
\definition{
    Just like a regular exterior algebra, there are natural notions of \textit{interior product} and \textit{Lie derivatives}, defined for $w \in \Gamma(\bigwedge^{k} A^{*})$ and $a, a_0, \ldots, a_{k-1} \in \Gamma(A)$
    as 
    \begin{align*}
        \iota_a \omega(a_0, \ldots, a_{k-1}) &= \omega(a, a_0, \ldots , a_{k-1})\\
        \mathcal{L}_a \omega(a_0, \ldots, a_{k_1}) &=
        \sum_{i=0}^k \omega(\ldots, a_{i-1}, [a, a_i], \ldots, a_{k-1}) - \rho_*a \omega(a_0, \ldots, a_{k-1}),
    \end{align*}
    with extension by linearity.
}\label{def:la_cartan_calculus}
% 
\fact{
These follow the usual Cartan calculus identity:
\begin{tcolorbox}
    \begin{align}
        \mathcal{L}_a &= \iota_a \circ d_A + d_A \circ \iota_a\\
        [\mathcal{L}_a, \mathcal{L}_b] &= \mathcal{L}_{[a,b]}\\
        [\mathcal{L}_a, \iota_b] &= \iota_{[a,b]}\\
        [\iota_a, \iota_b] &=0\\
        [\mathcal{L}_a, d] &= 0,
    \end{align}
    where we consider the commuttators on the graded algebra, so be carefull.
    So the operators $\iota, \mathcal{L}, d$ have respectively degrees $-1,0, 1$.
\end{tcolorbox}
}\label{fact:Cartan Calculus}

\proposition{
    There is a 1-1 correspondence between Lie algebroids and \textit{Linear Poisson structures}.
}\label{prop:correspondance_with_poisson_structures}

Maybe talk about splitting and covariant derivatives?

% 
% \definition{
%     Let P be an n-dimensional manifold and let $A^{k}(P) = \Gamma(\bigwedge^{k+1}TP)$. There exists a unique bracket $[\cdot,\cdot] : A^k(P)\times A^l(P) \rightarrow A^{k+l}(P)$ such that 
%     \begin{itemize}
%         \item $\forall X \in A^0(P) = \mathcal{X}(P)$, the bracket of vector fields (degree 0) is the Lie derivative $[X,\cdot] = \mathcal{L}_X$,
%         \item $\forall X \in A^k(P)\: \forall Y \in A^l(P)$, the graded antisymmetry:
%         $ [X,Y] = - (-1)^{kl}[Y,X]$,
%         \item $\forall X \in A^k(P)$, $[X, \cdot]$ is a derivation of degree k. 
%         \footnote{
%         recall that a derivation D of degree k has
%         $D(ab) = D(a)b + (-1)^{k |b|}aD(b)$. might be $|a|$ instead actually. CHECK
%         }
%     \end{itemize}
%     The \textbf{Schouten-Nijenhius} bracket is the unique extension of the Lie bracket to a $\mathbbm{Z}$-graded bracket on the space of forms.
%     }
% \begin{proof}[Proof of \cref{schouten_poisson}]
%     One needs only prove that the Poisson bracket $\{ , \} $ satisfies the Jacobi identity if and only if $\Pi$ has vanishing Schouten bracket to complete the proof that $(P, \Pi)$ defines a Poisson manifold.
% \end{proof}
% subsection algebraic_structures_associated_with_la (end)

% 
\subsection{Connections on Lie algebroids} % (fold)
\label{sub:connections_on_la}
A good ref is \citep{Lyakhovich2003}.
Can view the concept of a Lie algebroid as a tool to transfer the differential geometry of tangent bundles to abstract vector bundles.
% 
An important notion to grasp is the extension of the Koszul connection (on vector bundle) to algebroid-valued connection on a vector bundle. This is the so called A-connection defined below.
\definition{
    Given a Lie algebroid $A \rightarrow M$, a \textit{A-connection} $~^A\nabla$ \textbf{on} the tangent bundle $TM$ is a \textit{Kozul connection} on $TM$ that is compatible with the anchor map $\rho: A \rightarrow TM$ \citep{Kotov2019}.\\
    At the level of sections of $A$, the $A$-connection is defined as
    \begin{align*}
        ~^A\nabla: \Gamma(A) \times \Gamma(TM) &\rightarrow \Gamma(TM) \\
        (s, X) & \mapsto ~^A\nabla_s X 
    \end{align*}
    satisfying the Leibniz identity
    \begin{equation}
        ~^A\nabla_s ( f \cdot X) = \rho_{*}(s)(f) X + f ~^A \nabla_s X.
    \end{equation}
    Alternatively, in the Leibniz characterisation (see later in\cref{prop:Leibniz characterisation} ), the connection $~^A\nabla \in \mathrm{Diff}_1(TM, TM)$ is uniquely defined through maps $~^A \nabla \in \Gamma_\mathbb{R}(T^*M \otimes TM)$ and $\mathcal{C}^\infty(M)$-linear symbol map 
    $\delta : T^*M \rightarrow \Gamma(T^*M \otimes TM)$  by
    \begin{equation}
        \rho_*(s)(f) = \delta (df) (\rho_*(s))
    \end{equation}
}\label{def:algebroid valued connection on a vector space}

The compatibility with the anchor is integrated in the Leibniz identity, this is really the only equation we can write down. Below we explore other connections that can be induced on the algebroid itself and the tangent bundle. In some sense the $A$-connection intertwines between connections on an abstract vector bundle and a regular tangent bundle.
% \footnotetext{
%     delete part about the der bundle
%     Some subtleties I don't get here yet.            
% }
% 
\definition{
    The \textbf{curvature} $R_\nabla \in \Omega^2(A)$ of $A$-connection $\nabla$ of a Lie algebroid $A \rightarrow M$ 
    \begin{align*}
        R_\nabla : \Gamma(A) \times \Gamma(A) &\rightarrow \mathrm{End}(\Gamma(TM))\\
        R_\nabla (a,b) &= [\nabla_a, \nabla_b] - \nabla_{[a,b]}
    \end{align*}
    for $a,b \in \Gamma(A)$. 
    Alternatively, given $X \in \Gamma(TM)$, the curvature is a $2$-form $R_\nabla \in \Omega(A)$.
    If the curvature vanishes, we say that the connection is \textit{flat} and we call $(A, \nabla)$ a \textbf{representation} of the Lie algebroid over itself.
    This is because this definition can be extended where $A \rightarrow M$ a Lie algebroid and $E \rightarrow M$ a vector bundle and looking at morphisms between them. Fortunately we only need $A = E$.
}\label{def:Curvature of Lie algebroid connection}

You can play the same game with $\Omega_E^\bullet(A) = (\Gamma(\bigwedge^{\bullet} A^{*} \otimes E), \wedge, d_A^E)$ which will lead to \textit{Lie algebroid cohomology with coefficient in E}. This maybe what I need for my problem.
% subsection connections_on_la (end)
\subsection{Lie integration of Algebroids} % (fold)
\label{sub:lie_integration_of_algebroids}

% subsection lie_integration_of_algebroids (end)
% section lie_algebroids (end)