% !TeX root = ../main.tex
% 
\lhead{\emph{Lecture 1}}
\section{Lecture 1: Poisson and Presymplectic geometry} % (fold)
\label{sec:lecture_1_poisson_and_presymplectic_geometry}
\subsection{Poisson Algebra} % (fold)
\label{sub:poisson_algebra}

\definition{
    A \textbf{Poisson Algebra} is a triple $(A, \cdot, \{ , \})$ such that
    \begin{enumerate}
        \item $(A, \cdot)$ is a commuttative, associative and unital $\mathbbm{R}$-algebra (or $\mathbb{C}$ algebra maybe?)
        \item $(A, \{ , \})$ is a Lie $\mathbbm{R}$-algebra, which means that the bracket follows the Jacobi identity:
        \begin{equation}
            \{ \{ a, b \},c \} + \{ \{ b,c \},a \} + \{ \{ c,a \},b \} = 0 
        \end{equation}
        \item The Poisson bracket follows the Libeniz identity in the sense that for $a,b,c \in A$,
        \begin{align}
            \{a,b\cdot c\} &= \{a,b\} \cdot c + b\cdot \{a,c\}\\
            &:= \mathrm{ad}_{a} \, (b\cdot c)
        \end{align}
        where we have defined the adjoint map of the Lie algebra.
        \item Equivalently to $3$, the $\mathrm{ad}_{\{ , \}} : A \rightarrow \mathrm{Der}(A, \cdot)$, which takes an element of the algebra to a derivation on the commuttative algebra $(A, \cdot)$.
        We also see that the $\mathrm{ad}_{\{  \}}$ induces a derivation on $(A, \{ , \})$ using the Jacobi identity.
    \end{enumerate}
    Therefore the adjoint map of a Lie Algebra is a morphism from $A$ to $\mathrm{Der}(A, \cdot) \cap \mathrm{Der}(A, \{ , \})$, the derivations of both bilinear structures of a Poisson algebra.
}
\definition{
    A \textbf{Poisson derivation} is a derivation on both bilinear forms of a Poisson algebra, that is $X \in \mathrm{Der}(A, \cdot) \cap \mathrm{Der}(A, \{ , \}) \subset \mathrm{End}_\mathbb{R}(A)$.
    If a Poisson derivation is generated by the adjoint map, $X_a = \{ a, \}$, we say that it is a \textbf{Hamiltonian derivation}.
}\label{def:Poisson_hamilton}

\definition{
    A Poisson Algebra morphism is a linear map $\psi:A \rightarrow B$ such that
    $\psi : (A, \cdot) \rightarrow (B, \cdot)$ is an algebra morphism and
    $\psi : (A, \{ , \}) \rightarrow (B, \{ , \})$ is a Lie algebra morphism.
}

\definition{
    A subalgebra $I \subset A$ is \textbf{coisotrope} if
    \begin{itemize}
        \item $I \subset (A, \cdot)$ is a multiplicative ideal
        \item $I \subset (A, \{ , \})$ is a Lie subalgebra
    \end{itemize}
}

\proposition{\textit{Reduction of Poisson algebra}}{
    \\
    Suppose $I \subset A$ coisotrope and consider the Lie normaliser (or in ring theory the idealiser)
    \begin{align}
        N(I) = \{ a \in A | \{ a, I \} \subset I \},
    \end{align}
    which is the largest subalgebra of $A$ that contains $I$ as an ideal.
    We claim that $A' := \faktor{N(I)}{I} \,$ inherits a Poisson algebra structure.
}
\begin{proof}
    Condition 1 is automatically satisfied as $A'$ is a subalgebra of $A$, with a Lie algebra structure given by the same bracket. For $a',b',c' \in A'$, consider the adjoint action of $a'$ on $b'\cdot c'$ and look at coset representatirve $a,b,c$ of $N(I)$. Using the fact that $I$ is coisotrope, we see that
    \begin{align*}
        \{ a + I, (b +I)\cdot (c + I)\} &= \{ a+I, b\cdot c + I \}\\
        &= \{ a, b \cdot c \} + I
    \end{align*}
    by linearity of the bracket and closure of elements in $N(I)$ w.r.t $I$.
    The jacobi identity is checked by similar arguments.
\end{proof}
\definition{
    The \textit{reduced Poisson structure} is characterised by the projection map
    $ p: (N(I), \cdot, \{ , \} ) \rightarrow  (A', \cdot', \{ , \}' )$, and by the above proposition, this is a Poisson Algebra morphism.
}
% subsection poisson_algebra (end)
\subsection{Poisson Manifolds} % (fold)
\label{sub:poisson_manifolds}
\definition{
    A \textbf{Poisson manifold} is a smooth manifold P whose commutative algebra of smooth functions has the structure of a Poisson algebra $(\mathrm{C}^{\infty}(P), \cdot, \{ , \} )$.
}
\definition{
    A map $\phi: P_1 \rightarrow P_2$ is a \textit{Poisson map} if $\phi^{*}: \mathrm{C}^{\infty}(P_2) \rightarrow \mathrm{C}^{\infty}(P_1)$ is a Poisson morphism of algebras. 
}
% 
\begin{tcolorbox}
    Recall that derivations on smooth functions are isomorphic to vector fields:
    \begin{equation}
        \mathrm{Der}( \mathrm{C}^{\infty}(P)) \simeq \Gamma( TP),
    \end{equation}
    where the isomorphism is due to
    \begin{equation}
        \{ f,g \} \mapsto  X_{\{ f,g \} } = [X_f,X_g]
    \end{equation}
% 
\end{tcolorbox}
% 
\definition{
    So following through definition \cref{def:Poisson_hamilton}, the Poisson derivations on a Poisson manifolds are called \textbf{Poisson vector fields}. And Hamiltonian derivations on Poisson manifolds are called \textbf{Hamiltonian vector fields}.
    Hamiltonian vector fields are generated by the adjoint map
    \begin{align*}
        \mathrm{ad} : \mathrm{C}^{\infty}(P) &\rightarrow \Gamma(TP)   \\
        f &\mapsto X_f := \{ f, \cdot \} 
    \end{align*}
}
\proposition{
    A manifold $P$; with a commutative algebra of smooth functions $(\mathrm{C}^{\infty}(P), \cdot, \{ , \} )$, and a bivector $\Pi \in \Gamma( \bigwedge T^2P)$ defined as
    \begin{equation}
        \Pi(df,dg) = \{ f,g \}; 
    \end{equation}
    is a Poisson manifold if and only if $\Pi$ has vanishing Schouten bracket
    \begin{equation}
        \llbracket \Pi, \Pi \rrbracket = 0.
    \end{equation}
    Before proving this statement, we recall facts about the Schouten-Nijenhius which forms a special case of a \textit{Gerstenhaber algebra}. CHECK THIS!!
}\label{schouten_poisson}
\definition{
    Let P be an n-dimensional manifold and let $A^{k}(P) = \Gamma(\bigwedge^{k+1}TP)$. There exists a unique bracket $[\cdot,\cdot] : A^k(P)\times A^l(P) \rightarrow A^{k+l}(P)$ such that 
    \begin{itemize}
        \item $\forall X \in A^0(P) = \mathcal{X}(P)$, the bracket of vector fields (degree 0) is the Lie derivative $[X,\cdot] = \mathcal{L}_X$,
        \item $\forall X \in A^k(P)\: \forall Y \in A^l(P)$, the graded antisymmetry:
        $ [X,Y] = - (-1)^{kl}[Y,X]$,
        \item $\forall X \in A^k(P)$, $[X, \cdot]$ is a derivation of degree k. 
        \footnote{
        recall that a derivation D of degree k has
        $D(ab) = D(a)b + (-1)^{k |b|}aD(b)$.
        }
    \end{itemize}
    The \textbf{Schouten-Nijenhius} bracket is the unique extension of the Lie bracket to a $\mathbbm{Z}$-graded bracket on the space of forms.
    }
\begin{proof}[Proof of \cref{schouten_poisson}]
    One needs only prove that the Poisson bracket $\{ , \} $ satisfies the Jacobi identity if and only if $\Pi$ has vanishing Schouten bracket to complete the proof that $(P, \Pi)$ defines a Poisson manifold.
\end{proof}
% subsection poisson_manifolds (end)

COMPLETE LECTURE LATER
% section lecture_1_poisson_and_presymplectic_geometry (end)