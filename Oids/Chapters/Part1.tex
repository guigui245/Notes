% !TeX root = ../main.tex
% 
\lhead{\emph{Lecture 1}}
\section{Lecture 1: Poisson and Presymplectic geometry} % (fold)
\label{sec:lecture_1_poisson_and_presymplectic_geometry}
The first lecture is mostly based on section $2.4$ of \citep{Zapata-Carratala2019}.
\subsection{Poisson Algebra} % (fold)
\label{sub:poisson_algebra}

\definition{
    A \textbf{Poisson Algebra} is a triple $(A, \cdot, \{ , \})$ such that
    \begin{enumerate}
        \item $(A, \cdot)$ is a commuttative, associative and unital $\mathbbm{R}$-algebra (or $\mathbb{C}$ algebra maybe?)
        \item $(A, \{ , \})$ is a Lie $\mathbbm{R}$-algebra, which means that the bracket follows the Jacobi identity:
        \begin{equation}
            \{ \{ a, b \},c \} + \{ \{ b,c \},a \} + \{ \{ c,a \},b \} = 0 
        \end{equation}
        \item The Poisson bracket follows the Libeniz identity in the sense that for $a,b,c \in A$,
        \begin{align}
            \{a,b\cdot c\} &= \{a,b\} \cdot c + b\cdot \{a,c\}\\
            &:= \mathrm{ad}_{a} \, (b\cdot c)
        \end{align}
        where we have defined the adjoint map of the Lie algebra.
        \item Equivalently to $3$, the $\mathrm{ad}_{\{ , \}} : A \rightarrow \mathrm{Der}(A, \cdot)$, which takes an element of the algebra to a derivation on the commuttative algebra $(A, \cdot)$.
        We also see that the $\mathrm{ad}_{\{  \}}$ induces a derivation on $(A, \{ , \})$ using the Jacobi identity.
    \end{enumerate}
    Therefore the adjoint map of a Lie Algebra is a morphism from $A$ to $\mathrm{Der}(A, \cdot) \cap \mathrm{Der}(A, \{ , \})$, the derivations of both bilinear structures of a Poisson algebra.
}
\definition{
    A \textbf{Poisson derivation} is a derivation on both bilinear forms of a Poisson algebra, that is $X \in \mathrm{Der}(A, \cdot) \cap \mathrm{Der}(A, \{ , \}) \subset \mathrm{End}_\mathbb{R}(A)$.
    If a Poisson derivation is generated by the adjoint map, $X_a = \{ a, \}$, we say that it is a \textbf{Hamiltonian derivation}.
}\label{def:Poisson_hamilton}

\definition{
    A Poisson Algebra morphism is a linear map $\psi:A \rightarrow B$ such that
    $\psi : (A, \cdot) \rightarrow (B, \cdot)$ is an algebra morphism and
    $\psi : (A, \{ , \}) \rightarrow (B, \{ , \})$ is a Lie algebra morphism.
}

\definition{
    A subalgebra $I \subset A$ is \textbf{coisotrope} if
    \begin{itemize}
        \item $I \subset (A, \cdot)$ is a multiplicative ideal
        \item $I \subset (A, \{ , \})$ is a Lie subalgebra
    \end{itemize}
}\label{def:coisotrpoe_subalgebra}

\proposition{\textit{Reduction of Poisson algebra}}{
    \\
    Suppose $I \subset A$ coisotrope and consider the Lie normaliser (or in ring theory the idealiser)
    \begin{align}
        N(I) = \{ a \in A | \{ a, I \} \subset I \},
    \end{align}
    which is the largest subalgebra of $A$ that contains $I$ as an ideal.
    We claim that $A' := \faktor{N(I)}{I} \,$ inherits a Poisson algebra structure.
}
\begin{proof}
    Condition 1 is automatically satisfied as $A'$ is a subalgebra of $A$, with a Lie algebra structure given by the same bracket. For $a',b',c' \in A'$, consider the adjoint action of $a'$ on $b'\cdot c'$ and look at coset representatirve $a,b,c$ of $N(I)$. Using the fact that $I$ is coisotrope, we see that
    \begin{align*}
        \{ a + I, (b +I)\cdot (c + I)\} &= \{ a+I, b\cdot c + I \}\\
        &= \{ a, b \cdot c \} + I
    \end{align*}
    by linearity of the bracket and closure of elements in $N(I)$ w.r.t $I$.
    The jacobi identity is checked by similar arguments.
\end{proof}
\definition{
    The \textit{reduced Poisson structure} is characterised by the projection map
    $ p: (N(I), \cdot, \{ , \} ) \rightarrow  (A', \cdot', \{ , \}' )$, and by the above proposition, this is a Poisson Algebra morphism.
}
% subsection poisson_algebra (end)
\subsection{Poisson Manifolds} % (fold)
\label{sub:poisson_manifolds}
\definition{
    A \textbf{Poisson manifold} is a smooth manifold P whose commutative algebra of smooth functions has the structure of a Poisson algebra $(\mathrm{C}^{\infty}(P), \cdot, \{ , \} )$.
}
\definition{
    A map $\phi: P_1 \rightarrow P_2$ is a \textbf{Poisson map} if $\phi^{*}: \mathrm{C}^{\infty}(P_2) \rightarrow \mathrm{C}^{\infty}(P_1)$ is a Poisson morphism of algebras. 
}
% 
\begin{tcolorbox}
    Recall that derivations on smooth functions are isomorphic to vector fields:
    \begin{equation}
        \mathrm{Der}( \mathrm{C}^{\infty}(P)) \simeq \Gamma( TP),
    \end{equation}
    where the isomorphism is due to
    \begin{equation}
        \{ f,g \} \mapsto  X_{\{ f,g \} } = [X_f,X_g]
    \end{equation}
% 
\end{tcolorbox}
% 
\definition{
    So following through definition \cref{def:Poisson_hamilton}, the Poisson derivations on a Poisson manifolds are called \textbf{Poisson vector fields}. And Hamiltonian derivations on Poisson manifolds are called \textbf{Hamiltonian vector fields}.
    Hamiltonian vector fields are generated by the adjoint map
    \begin{align*}
        \mathrm{ad} : \mathrm{C}^{\infty}(P) &\rightarrow \Gamma(TP)   \\
        f &\mapsto X_f := \{ f, \cdot \} 
    \end{align*}
}
\proposition{
    A manifold $P$; with a commutative algebra of smooth functions $(\mathrm{C}^{\infty}(P), \cdot, \{ , \} )$, and a bivector $\Pi \in \Gamma( \bigwedge^2 TP)$ defined as
    \begin{equation}
        \Pi(df,dg) = \{ f,g \}; 
    \end{equation}
    is a Poisson manifold if and only if $\Pi$ has vanishing Schouten bracket
    \begin{equation}
        \llbracket \Pi, \Pi \rrbracket = 0.
    \end{equation}
    Before proving this statement, we recall facts about the Schouten-Nijenhius which forms a special case of a \textit{Gerstenhaber algebra}. We will define Gerstenhaber algebras in \cref{def:Gerstenhaber}.
}\label{schouten_poisson}
% 
\begin{proof}
    The Schouten-Nijenhius bracket is defined as a degree $-1$ bracket on the differential graded algebra of alternating multivector fields, so
    \begin{equation}
        \llbracket \cdot, \cdot \rrbracket : \Gamma( \bigwedge^2 TP) \times \Gamma( \bigwedge^2 TP) \rightarrow \Gamma( \bigwedge^3 TP).
    \end{equation}
    By considering $f,g,h \in C^\infty(P)$ with corresponding forms $\alpha,\beta, \gamma \in  \Omega^1(P)$, then
    \begin{align*}
        \llbracket \Pi, \Pi\rrbracket(\alpha, \beta, \gamma) &= \Pi( \Pi(\alpha, \beta), \gamma) + \mathrm{cyclic} \\
        \Leftrightarrow \qquad \qquad  &= \left\{\left\{  f,g\right\}, h  \right\} 
        + \left\{ \left\{ g,h \right\},f \right\}
        + \left\{  \left\{ h,f \right\},g \right\} \\
        \Leftrightarrow \qquad \qquad &=0 \quad \quad \mathrm{(Poisson)}
    \end{align*}
    
\end{proof}

% 
\definition{
    Given a Poisson bivector $\Pi$, the musical map (sharp) 
    \begin{align}
        \Pi^\sharp &: T^*P \rightarrow TP \\
        df &\mapsto \Pi (df, \cdot) := \{f, \cdot\}
    \end{align}
    defines an \textbf{Hamiltonian distribution}.
    Equivalently, 
    \begin{equation*}
        X_{\cdot} = \Pi^\sharp \circ d : C^\infty(P) \rightarrow \Gamma(TP) 
    \end{equation*}
    is an \textit{Hamiltonian map}. Note that the space of Hamiltonian distribution $\Pi^\sharp(T^*P)$ is involutive as it is a Lie algebra morphism.
}\label{def:hamiltonian_distribution}

\definition{
    A submanifold $C \subset (P, \Pi)$ is \textbf{coisotropic} if $TC \subset (TP, \Pi)$ is a coisotropic subspace, that is $TC \supset (TC)^0$ an isotropic (sometimes denoted $TC \supset TC^\perp$) subspace of $TC$ with respect to the bivector:
    \begin{equation}
        \Pi(\alpha, \beta) = 0 \quad \forall \alpha \in (T^*C)^0, \forall \beta \in T^*C
    \end{equation}
    % \begin{equation*}
    %     \forall \alpha \in (T^*C)^0 \, , \forall v \in TP \quad \alpha(v) = 0.
    % \end{equation*}
    Consequently, the short exact sequence:
    \begin{equation}
        0 \rightarrow (T^*C)^0 \xrightarrow{\Pi^\sharp} TC \rightarrow C^\infty(C)\rightarrow 0
    \footnote{
        i think this is right, but not sure
    }
    \end{equation}
}\label{def:coisotropic_submanifold}

It is useful to make the connection between geometry and algebra, relating manifolds to ideals of the algebra of functions on said manifold.
In \citep{OFarrill1989}, a nice treatise is presented in the following.
% 
% 
\remark{
    It is useful to study manifolds $M$ in a `dual' way by considering the commutative algebra of smooth functions $C^\infty(M)$. There exits as isomorphism between $M$ and the set $\mathcal{M}$ of all maximal ideal $I(p)$ consisting of functions $f \in C^\infty(M)$ such that $f\big|_p = 0$.
}\label{rem:dual_notion_mfd}
\proposition{
    Let $\iota : C \hookrightarrow P$ be a closed submanifold of Poisson manifold $(P, \Pi)$, then the following are equivalent:
    \begin{itemize}
        \item C is coisotropic
        \item The vanishing ideal $I_C = \mathrm{ker}(\iota^*) := \{ g \in C^\infty(P)\,  \Big\vert \:  g\rvert_C = 0\}$ is a coisotrope of the Poisson algebra $(C^\infty(P), \cdot, \{\cdot, \cdot\})$.
        \item Hamiltonian vector fields $X_g$ generated by $g \in I_C$ are tangent to $C$: $X_g|_C \in \Gamma(TC)$ 
    \end{itemize}
% 
    \begin{proof}
        \begin{itemize}
            \item $(1) \Rightarrow (2)$: First $(I_C, \cdot)$ is a multiplicative ideal of $(C^\infty(P), \cdot)$ by construction. 
            Further, if $f,g \in I_C$ then $df,dg \in (T^*C)^\perp$ and the associated Poisson bracket vanishes, making $I_C$ into a Lie subalgebra.
            So $I_C$ is coisotropic to the Poisson Algebra on $P$. 
            % 
            \item $(2) \Rightarrow (3)$: for a basis $g \in I_C$, the Hamiltonian vector fields $X_g = \{g, \cdot\}$ span the $\mathrm{Der}(C^\infty(C))$ which is the space of tangent vector to $C$.
            % 
            \item $(3) \Rightarrow (1)$: 
            $\iota^* \{g,f\} = 0$  for $g \in I_C$, $\forall f \in C^\infty(P)$ \footnote{
                continue later
            }
        \end{itemize}
    \end{proof}
}\label{prop:characterisation_of_coisotropic_submfd}

\definition{
    Consider $2$ Poisson manifold $(P_1, \Pi_1)$ and $(P_2, \Pi_2)$, the \textit{product Poisson manifold} is $(P_1 \times P_2, \Pi_1 + \Pi_2)$, where the canonical isomorphism 
    $T(P_1 \times P_2) \cong \mathrm{pr}_1^* TP_1 \oplus \mathrm{pr}_2^* TP_2$. \\
    The Whitney sum of vector bundle $A_M,B_M$ over manifold $P$ is defined as above by
    \begin{equation}
        A_M  \boxplus B_M = \mathrm{pr}_1^* A_M \oplus \mathrm{pr}_2^* B_M
    \end{equation}
    % 
    Also it's easy to see that pulling back onto either $P_1,P_2$ commutes with the bracket structure, with "cross-pulling" bracket vanishing
}\label{def:product_poisson_mfd}

\definition{
    Given a Poisson manifold $(P, \Pi)$, \textbf{opposite Poisson manifold} is $\overline{P} = (P, - \Pi)$.
}\label{def:opposite_poisson_mfd}

\proposition{
    Let two Poisson manifold $(P_1, \Pi_1),(P_2, \Pi_2)$ and a smooth map $\phi: P_1 \rightarrow P_2$, then $\phi$ is a Poisson map if and only if
    \begin{equation*}
        \mathrm{grph}(\phi):= \{(p, \phi(p)) \;| \; \forall p \in P_1\} \subset P_1 \times \overline{P}_2
    \end{equation*}
    is a coisotropic submanifold.
    \begin{proof}
        Consider the tangent bundle of the graph submanifold 
        \begin{equation*}
        T \mathrm{grph}(\phi) = \{(X, Y)\: |\: \textrm{if } \exists Y \in TP_2 \textrm{ such that } X,Y \textrm{are } \phi \textrm{-related: }   \phi^*Y = \phi_*X \}.
        \end{equation*}
        Full proof in \citep{Fernandes2015} but they have a weird definition of $\Pi^\sharp$ there. 
        \footnote{
            continue one day
        }
        % \begin{itemize}
        %     \item Let $\phi$ be a Poisson map, that is $\phi^*: C^\infty(P_2) \rightarrow C^\infty(P_1)$ a Poisson morphism of algebras then $\phi^*f$
        % \end{itemize}
    \end{proof}
}\label{prop:coisotropic_relation}

\begin{tcolorbox}
    Recall that a \textit{submersion} is a differential map $\phi : M \rightarrow N$ such that
    \begin{equation}
        D \phi_p : T_pM \twoheadrightarrow T_{\phi_p}N
    \end{equation}
    for all $p \in M$.
    Dually, an \textit{immersion} is a differential map $\phi : M \rightarrow N$ such that
    \begin{equation}
        D \phi_p : T_pM \hookrightarrow T_{\phi_p}M
    \end{equation}
    for all $p \in M$.
\end{tcolorbox}
We now consider the important notion of coisotropic reduction. A full treatment is given in \citep{Marsden1986}, but we focus on the simpler case where the involutive distribution on $C \hookrightarrow P$ is given by the sets of its Hamiltonian vector fields.

\proposition{(Coisotropic Reduction of Poisson manifold)\\
    Let $(P, \Pi)$ a Poisson manifold, and $\iota: C \hookrightarrow P$ a closed coisotropic submanifold. 
    Let $\{X_{I_C}\}$ be the set of Hamiltonian vector field tangent to $C$ generated by the ideal $I_C = \mathrm{ker}(\iota^*)$.
    This integrates to a regular foliation $\chi_C$ on C because of the involution
    \begin{equation}
        [X_{I_C}, X_{I_C}] \subset X_{\left\{ I_C, I_C \right\}} \subset X_{I_C}.
    \end{equation}
    Further assume that the leaf space is smooth $P':= \faktor{C}{\chi_C}$ such that there is a submersion (quotient map) $q$ fitting in the reductive diagram
    \begin{equation}
        \begin{tikzcd}
            C \arrow[d, twoheadrightarrow, "q"]\arrow[r, hook, "\iota"] & (P, \Pi) \\
            (P', \Pi')
        \end{tikzcd}
    \end{equation}\\
    Then $P'$ inherits a Poisson structure on functions $(C^\infty(P'), \{\cdot, \cdot\}')$ that is uniquely determined by the condition
    \begin{equation}
        \iota^* \{F,G\} = q^* \{f,g\}'
    \end{equation}
    for all $f,g \in C^\infty(P')$ and $F,G \in C^\infty(P)$ such that $F,G$ are the leaf-wise constant extensions of $f,g$, i.e
    \begin{align*}
        q^*f &= \iota^*F\\
        q^*g &= \iota^*g
    \end{align*}
}\label{prop:coisotropic_reduction}
\begin{proof}
    Following \citep{Marsden1986}, theorem 2.2.
    As opposed to the theorem in the paper, we consider the trivial case where the involutive distribution is given by the set of Hamiltonian vector fields $X_{I_C}$. 
    It's clear from \cref{prop:characterisation_of_coisotropic_submfd} that on $C$, the distribution of Hamiltonian vector field vanishes: $X_{I_C} = 0 \subset TC$.
\end{proof}

The upshot is, we have identified a Poisson submanifold $(P', \Pi')$ with a reduced Poisson structure, all this is because of the coisotropic datum given.

\remark{
    Let $\tilde P \subset P$ a Poisson submanifold, then $\Pi^\sharp (T^* \tilde P) = X_{I_{\tilde P}} =0$
    is equivalent to  $\tilde P \hookrightarrow P$ is a Poisson morphism.
}
\example{
    Consider a Lie group action $G \circlearrowright (P, \Pi)$ via the Poisson map $G \times P \rightarrow P$ that is also a morphism of Poisson algebras. 
    Infinitesimally, this is the action of $\mathfrak{g} = \mathrm{Lie}(G)$ on $P$ by the Lie algebra-valued Hamiltonian vector fields given by the map $\psi$ and the comoment map $\overline{\mu}$ defined as
    \begin{align*}
    \psi: \mathfrak{g} &\rightarrow \Gamma(TP)\\
    \overline{\mu} : \mathfrak{g} & \rightarrow C^\infty(P)
    \end{align*}
    such that, for all $\xi,\zeta \in \mathfrak{g}$
    \begin{align}
        \psi(\xi) &= X_{\overline{\mu}(\xi)} \coloneq \left\{ \overline{\mu}(\xi), \cdot \right\}    \\
        \overline{\mu}([\xi, \zeta]) &= \left\{ \overline{\mu}(\xi), \overline{\mu}(\zeta) \right\}
    \end{align}
    The map $\overline{\mu}$ is called the \textbf{comoment map} and it is a Lie algebra morphism between the Lie algebra $\mathfrak{g}$ and the Poisson algebra.
    Dually, we define the \textbf{moment map} such that the following diagram commutes
    \begin{equation}
        \begin{tikzcd}
            g \arrow["\overline{\mu}"]{r}&C^\infty(P)\\
            g^* \arrow["\mathrm{Hom}"]{u}&P \arrow[]{u} \arrow["\mu"]{l}
        \end{tikzcd}
    \end{equation}
    or by the relation
    \begin{equation}
        \langle \mu(p), \xi \rangle = X_{\overline{\mu}(\xi)}(p)
    \end{equation}
    for $p \in P$ and $\xi \in \mathfrak{g}$.\\
    % 
    If $0 \in \mathfrak{g}^*$ is a regular value, then $C \coloneq \mathrm{ker}(\mu) \subset P$ is a coisotropic submanifold because the equation above defines a tangent distribution $X_{\overline{\mu}(\xi)} = 0$ for all $\xi \in \mathfrak{g}$. Therefore by \cref{prop:characterisation_of_coisotropic_submfd}, we have a coisotropic submanifold.
    % 
    If the Poisson action is free and proper, then $\mathrm{ker}(\mu)/G$ is the coisotropic reduction of $(P, \Pi)$, and in this case this is an \textit{Hamiltonian reduction}.
    If the quotient space is not a manifold, then we can still construct a groupoid action and reduction.
}\label{ex:Hamiltonian group action}


% subsection poisson_manifolds (end)
\subsection{Presymplectic manifold} % (fold)
\label{sub:presymplectic_mfd}
\definition{
    A \textbf{presymplectic manifold} $(S, \omega)$ is a smooth manifold $M$ and a closed $2-$form $\omega \in \Omega^2(S) \cong \Gamma( \bigwedge^2 T^*S)$ with $d \omega = 0$.
    We also say that $S$ carries a symplectic structure $\omega$.\\
    If the symplectic structure $\omega$ is exact, we say that $S$ has an exact presymplectic structure.
}\label{def:presymplectic manifold}

\definition{
    A smooth map $\phi: S_1 \rightarrow S_2$ between presymplectic manifold is a \textit{presymplectic map} if
    \begin{equation*}
         \phi^*\omega_2 = \omega_1.
    \end{equation*} 
}\label{def:presymplectic map}

\definition{
    A diffeomorphism $\phi: S_1 \rightarrow S_2$ that is also a presymplectic is called a symplectomorphism. Infinitesimally, this is generated by vector fields $X \in \Gamma(TS_1)$ that respect the symplectic form on $S_1$, 
    \begin{equation}
        \mathcal{L}_X \omega_1 = 0
    \end{equation}
}\label{def:symplectomorphism}

\definition{
    Given a presymplectic manifold $(S,\omega)$, we define the flat \textit{musical map} as
    \begin{align*}
        \omega^\flat &: TS \rightarrow T^*S\\
        X & \mapsto \omega(X, \cdot) \coloneq \iota_X \omega
    \end{align*}
    
}\label{def:symplectic musical map}
% subsection presymplectic_mfd (end)
\subsection{Symplectic manifold} % (fold)
\label{sub:symplectic_mfd}

% subsection symplectic_mfd (end)
% section lecture_1_poisson_and_presymplectic_geometry (end)