% !TeX root = ../main.tex
% 
\newpage
\lhead{\emph{Lecture 6}}
\section{Lecture 6: Local Lie Algebras} % (fold)
\label{sec:lecture_6_local_lie_algebras}
% 
We will now explore the important notion of \textbf{locality}, which we formalise from the physical intuition that things should only depend on local variables, or that an open neighbourhood around a point should be sufficient to reconstruct sections (onto whatever) at that point.
% 
\subsection{General Local Lie Algebras} % (fold)
\label{sub:general_local_lie_algebras}
% 
\definition{
    The \textbf{support} of a map $f : X \rightarrow Y$ is
    \begin{equation}
        \mathrm{supp}(f) = \left\{ x \in X \big | x \notin \mathrm{ker}(f)  \right\}
    \end{equation}
}\label{def:support}

\begin{tcolorbox}
    discuss support of a module at some point?
\end{tcolorbox}
\definition{
    A structure is \textbf{local}, if for any two maps $f,g: X \rightarrow Y$
    \begin{equation}
        \mathrm{supp}(fg) \subset \mathrm{supp}(f) \cap \mathrm{supp}(g)
    \end{equation}
    and the support is compatible with the structure.
}\label{def:locality}
% 
Some examples to illuminate what we mean:
\begin{itemize}
    \item The algebra of smooth functions on manifold $M$ has,
    \begin{equation*}
        \mathrm{supp}(fg) \subset \mathrm{supp}(f) \cap \mathrm{supp}(g)
    \end{equation*}
    for $f,g \in C^\infty(M)$.
    \item Similarly for a vector bundle $A \rightarrow M$,
    \begin{equation*}
        \mathrm{supp}(f \cdot s  ) \subset \mathrm{supp}(f) \cap \mathrm{supp}(s)
    \end{equation*}
    for $f \in C^\infty(M), s \in \Gamma(A)$.
    \item The Poisson bracket in a Poisson algebra as seen in \cref{def:Poisson algebra} has manifestly the same property since it is an algebra over smooth functions.
\end{itemize}
% 
\definition{
    A vector bundle $A \rightarrow M$ is a \textbf{local Lie Algebra} if the $\mathbb{R}$-linear bracket on the smooth sections is local, i.e
    \begin{equation}
        \mathrm{supp}([a,b]) \subset \mathrm{supp}(a) \cap \mathrm{supp}(b)
    \end{equation}
    for $a,b \in \Gamma(A)$.
}\label{def:local lie algebra-historical}
% 
Regarding the Lie bracket as a differential operator on smooth sections above prompts us to extend our the historical definition of local Lie algebra to vector bundles with a local differential structure.
% 
\definition{
    A vector bundle $A \rightarrow M$ is said to carry a local lie algebra \textit{structure} on its sections $\Gamma(A)$ if the lie bracket on the space of sections is $\mathbbm{R}$-linear and the adjoint map is a differential operator of degree 1.
    \begin{align*}
        \mathrm{ad}_{\cdot} : \Gamma(A) \rightarrow \mathrm{Diff}_1(A)
    \end{align*}
    Or in other words, $\mathrm{ad}_\cdot \in \mathrm{Diff}(A, J^1 A^* \otimes A)$, where $J^k A^* \otimes A$ is algebroid-valued k-jet on its dual using \cref{prop: differential operators equivalent to jets on vector bundles}.
}\label{def:local lie algebra-differential}
% 
\example{
    A first example of a local Lie algebra is the algebra of first order differential operators $\mathrm{Diff}_1(A)$ over a vector bundle $A \rightarrow M$.
    Consider the operators; using the Leibniz characterisation (\cref{prop:Leibniz characterisation}), $\Delta, \nabla \in \mathrm{Diff}_1(A)$ with respective symbol maps $\delta_\Delta, \delta_\nabla \in \Gamma(T^*M \otimes \mathrm{End}(A))$, where
    \begin{equation}
        \mathrm{Diff}_1(A) = \left\{ \Delta \in \mathrm{End}_{\mathcal{C}^\infty(M)}(\Gamma(A))\: \vert \:
        c_{a_0} \circ c_{a_1} (\Delta) = 0 \; \forall  a_i \in \mathcal{C}^\infty(M)
        \right\}
    \end{equation}
    In a similar fashion to \cref{lem:commutatot_of_diff_op}, we have that $[\Delta, \nabla] \in \mathrm{Diff}_1(A)$ from
    \begin{align*}
        [\Delta, \nabla](f \cdot s) &= \Delta \bigg( f \cdot \nabla (s) + \delta_\nabla(df)(s)\bigg) 
        - \nabla \bigg(
        f \cdot \Delta(s) + \delta_\Delta(df)(s)
        \bigg)\\
        &= f \cdot [\Delta, \nabla](s) + 
        \bigg(
        [\delta_\Delta(df), \nabla] + [\Delta, \delta_\nabla(df)]
        \bigg)(s)
    \end{align*}
    In the second line above, we have used the definition of the symbol map from \cref{eq:symbol map}, $\delta_\Delta(df) = \Delta(f) = [\Delta, f]$.
    Rewriting the last equation in terms of the symbol of the commutator $[\Delta, \nabla]$, we find
    \begin{equation}
        [\Delta, \nabla](f \cdot s) = f \cdot [\Delta, \nabla](s) + \delta_{\Delta \nabla}(df)(s).
    \end{equation}
    Therefore $[\Delta,\nabla] \in \mathrm{Diff}_1(A) \Leftrightarrow \delta_{\Delta \nabla}(df)$ is $\mathcal{C}^\infty(M)$-linear.
    We can further check that for this map to be linear, leads to another messy equation that measures the failure of differential operators to close under commutation. (as expected)
}\label{ex:Differential operators as local Lie algebra}
% 
% subsection general_local_lie_algebras (end)
\subsection{Derivative Lie algebras} % (fold)
\label{sub:derivative_lie_algebras}
% 
\theorem{
    test
}\label{thm:symbol_squiggle}
% subsection derivative_lie_algebras (end)
% section lecture_6_local_lie_algebras (end)