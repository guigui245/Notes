% !TeX root = ../main.tex
% 
\newpage
\lhead{\emph{Lecture 5}}
\section{Lecture 5: Differential Operators II} % (fold)
\label{sec:lecture_5_differential_operators_ii}
% 
\subsection{Differential operators on Manifolds} % (fold)
\label{sub:differential_operators_on_manifolds}
In this section, we will build up a correspondence between the algebraic theory and the geometric theory of differential operators on manifolds.
Throughout this lecture, we assume the algebra $\mathcal{A}$ to be unital, associative and commutative over the real, and we'll restrict very soon to the algebra of smooth functions over a manifold.

\definition{
    A \textit{point} $x$ is an algebra morphism $x: \mathcal{A} \rightarrow \mathbb{R}$. The dual $|\mathcal{A}|$ of the algebra $\mathcal{A}$ is the set of all such algebra morphisms.
}
\definition{
    The set of $\mathbb{R}$-valued functions on $|\mathcal{A}|$ is denoted
    \begin{equation}
        \tilde{\mathcal{A}} = \left\{ f_a : |\mathcal{A}| \rightarrow \mathbb{R} \quad \big\vert \quad f_a(x) = x(a) \: \forall a \in \mathcal{A} \right\}
    \end{equation}
}
\definition{
    An algebra $\mathcal{A}$ is said to be \textit{geometric} if
    \begin{equation}
        \bigcap_{x \in |\mathcal{A}|} \mathrm{ker}(x) = 0.
    \end{equation}
    The named geometric comes from the canonical isomorphism 
    \begin{equation}
        \mathcal{A} \cong \tilde{\mathcal{A}},
    \end{equation}
    therefore putting an algebra structure on the $\mathbb{R}$-valued functions on the set $|\mathcal{A}|$.
    This is an identification between geometric algebras and sets.
}
    It is possible to put a topology on such spaces, or to find a smooth structure in order to define the notion of continuous/smooth algebras.

Let's consider a smooth manifold $M$ and its ring of smooth functions $C^\infty(M)$, then the ring of regular differential operators $\mathrm{Diff}(M)$ has the $2$ equivalent definitions from \cref{thm:equiv_def} 
\footnote{
    I'm assuming here that the spectrum of the ring of smooth functions is non-singular...    
}:
\begin{align}
    \mathrm{Diff}(M) = \bigcup_p D^p(C^\infty(M))
\end{align}
We sometimes denote $\mathrm{Diff}_p(M) = D^p(C^\infty(M))$ for simplicity.
\footnote{
    Or even $\mathrm{Diff}(M, \mathbb{R})$ more precisely.
}
% subsection differential_operators_on_manifolds (end)
% 

A useful equivalence between elements of the algebra of smooth functions on a manifold, relating to differential operators is the following local relation.
% 
\definition{
    Let $f \in \mathcal{C}^\infty(M)$, the k-jet of $f$ at $x_0 \in M$ is the a coordinate-free truncation of the Taylor expansion of to k-th order.
    \begin{equation}
        (j^k_{x_0} f)(z) = \sum_{i=0}^k \frac{\partial^i}{\partial x^i} f(x_0) \cdot \frac{z^i}{i!}
    \end{equation}
}\label{def:jet}
We can define an equivalence relation of jets in the algebra of smooth functions.
For $f,g \in C^\infty(M)$
\begin{align*}
    f &\sim_x^k g\\
    \Leftrightarrow \qquad f(x) &= g(x), \quad \partial^{|i|} f (x) = \partial^{|i|} g(x) \textrm{ for } i \leq k
\end{align*}
where $|k|$ is the order of the differential operator (potentially the sum of integers).
This equivalence relation can be globalised into a vector bundle.
\definition{
    The \textbf{k-jet} bundle of a manifold $M$ is
    \begin{equation}
        J^k M := \bigcup_{x\in M} J^k_x M, \qquad J^k_x := \left\{ [f]^k_x , f\in C^\infty(M) \right\}
    \end{equation}
    under the equivalence relation defined above.
    There is a natural projective resolution
    \begin{equation}
        \begin{tikzcd}
            \ldots \xrightarrow{\pi^{k+1}} J^k M \xrightarrow{\pi^k} J^{k-1} M \rightarrow
            \ldots \xrightarrow{\pi^1} C^\infty(M) \rightarrow 0,
        \end{tikzcd}
    \end{equation}
    where the projective maps $\pi^k$ sends classes of functions in the $k$-jet to the class of function that agree on a lower order derivative, i.e: $\pi^{n-1} \circ \pi^n = 0$.
    There exists also a map
    \begin{align*}
        j^k: C^\infty(M) &\rightarrow \Gamma(J^k M)\\
        f &\mapsto j^k f(x) := [f]^k_x
    \end{align*}
    sending functions to their $k-$jet by just truncation of the Taylor expansion.
}\label{def:jet bundle for manifold}
% 
\proposition{
    \begin{equation}
        \mathrm{Diff}_k(M) \cong \Gamma((J^k M )^*)
    \end{equation}
}
\begin{proof}{(Sketch)\\}
    Let $\Delta \in \mathrm{Diff}_k(M) \subset \mathrm{End}(C^\infty(M))$ an $\mathbb{R}$-linear endomorphism then, by the universal property, there exists a $C^\infty(M)$-linear map $\delta : J^k M \rightarrow \mathbb{R}$ (in other words, the dual jet bundle) fitting the diagram
    \begin{equation}
        \begin{tikzcd}
            C^\infty(M) \arrow["j^k_x"]{r} \arrow["\Delta"]{d}& J^k_x M \arrow["\delta"]{d} \\
            C^\infty(M) & \mathbb{R} \arrow["\delta_*"]{l}
        \end{tikzcd}
    \end{equation}
    Meaning that at the level of sections, we have 
    \begin{equation}
        \Delta = \delta_* \circ j^k
    \end{equation}
    where the pushforward $\delta_*: \Gamma(J^k M ^*) \rightarrow C^\infty(M)$ makes the diagram commute.
\end{proof}
% 

\example{
    For example, consider a first order differential operator treated as a regular differential operator over the ring of smooth functions, 
    \begin{equation}
        \Delta \in \mathrm{Diff}_1(M) = \{ \Delta \in \mathrm{End}_\mathbb{C}(\mathcal{C}^\infty(M)) \quad \big \vert \quad \Delta a - a  \Delta =  \Delta(a) \in D^{0}(\mathcal{C}^\infty(M)) \quad \forall a \in \mathcal{C}^\infty(M) \}.
    \end{equation}
    Then 
\begin{tcolorbox}
    Really not sure here, but this all relates to the fact that the jet of a functions factorises and that $\delta$ is a derivation.
\end{tcolorbox}
}
\subsection{Differential operators on Vector Bundles} % (fold)
\label{sub:differential_operators_on_vector_bundles}
% 
\definition{
    Let $\mathcal{A}$ be a $\mathbb{K}$-algebra and $\mathcal{P}, \mathcal{Q}$ be $\mathcal{A}$-modules with
    \begin{align*}
        \mu_\mathcal{P}: \mathcal{A} \times \mathcal{P} &\rightarrow \mathcal{P}\\
        \mu_\mathcal{Q}: \mathcal{A} \times \mathcal{Q} &\rightarrow \mathcal{Q}
    \end{align*}
    their respective operation.
    For $\phi \in \mathrm{Hom}_\mathbb{K}(\mathcal{P}, \mathcal{Q})$ and $a \in \mathcal{A}$, we define the \textbf{commutator} of $\mathbb{K}$-linear morphisms
    \begin{align}
        c_a(\phi) &\in \mathrm{Hom}_\mathbb{K}(\mathcal{P},\mathcal{Q}) \nonumber\\
        c_a(\phi) &:= [\phi, a] = \phi \circ \mu_\mathcal{P}(a) - \mu_\mathcal{Q}(a) \circ \phi
    \end{align}
}
\definition{
    The set of \textbf{Differential Operator} of order less than $n$  \textit{between the $\mathcal{A}$-modules $\mathcal{P}, \mathcal{Q}$} is
    \begin{equation}
        D^n(\mathcal{P}, \mathcal{Q}) := \left\{ 
        \Delta \in \mathrm{Hom}_\mathbb{K}(\mathcal{P}, \mathcal{Q}) \:\big \vert \:
        c_{a_0} \circ \ldots \circ c_{a_n} (\Delta) = 0 \quad \forall a_i \in \mathcal{A}
         \right\}
    \end{equation}
}
Making the connection with regular differential operators on an algebra $A$ over the complex numbers in\cref{eq:regular differential operator}, the definition above is equivalent to the previous statement using $\theta a - a \theta = \theta(a) \equiv c_a(\theta) $.

We previously identified geometric/continuous/smooth algebras with sets/topological spaces/manifolds. 
Similarly, there is an equivalence between vector bundles and finitely generated projective modules over smooth algebras.
Building upon the previous definition, we define what a differential operator on vector bundle means.
% 
\definition{
    A \textbf{Differential Operator}  \textit{between vector bundles $A,B \rightarrow M$} is defined as
    \begin{equation}
        \mathrm{Diff}(A,B) := \bigcup_k \mathrm{Diff}_k(A,B),
    \end{equation}
    where $\mathrm{Diff}_k(A,B) := D^k(\Gamma(A), \Gamma(B))$ is the set of k-differential operators on sections of vector bundles, treated as $\mathcal{C}^\infty(M)$-modules.
}\label{def:differential operators between two vector bundles}
% 
\proposition{
    As we have seen, differential operators form a filtered structure and we now show that this gives the \textbf{symbol short exact sequence} of $\mathcal{C}^\infty(M)$-modules.
    \begin{equation}
           0\rightarrow \mathrm{Diff}_{k-1}(A,B)\rightarrow \mathrm{Diff}_k(A,B)  \xrightarrow{\sigma}
           \Gamma(\bigodot^k TM \otimes A^* \otimes B)  \rightarrow 0
           \label{eq:symbol ses}
    \end{equation}
    \begin{proof} 
        Note that since vector bundles are commutative algebras that respect the Jacobi identity, we have 
        \begin{equation}
            c_f \circ c_g = c_g \circ c_f
        \end{equation}
        for $f,g \in  C^\infty(M)$. 
        The map $\sigma$ is called the \textbf{symbol} and completes this sequence in the following way. 
        For $f_i \in C^\infty(M)$,
        \begin{align}
            \sigma: \Delta &\mapsto \sigma_\Delta \\
            \sigma_\Delta(df_1,\ldots,df_k) &:= c_{f_1} \circ \ldots \circ c_{f_k} (\Delta)
            \label{eq:symbol map}
        \end{align}
        We prove this is a multiderivation by considering the case $k=1$ and extending linearly.
        Acting on $a \in \Gamma(A)$, for $df \in T^*M$, 
        \begin{align*}
            \sigma_\Delta(df) (a) &= [\Delta, f] (a)\\
            &= \Delta (f \cdot a) - f \cdot \Delta(a) \in \Gamma(B)
        \end{align*}
        Now suppose $\Delta \in \mathrm{Diff}_0(A,B)$, then $c_{a}(\Delta) = 0 $ for all $a \in \Gamma(A)$, and so the image of $\Delta$ in the set of differential operators (between vector bundle) of order 1, is the kernel of the symbol map.\\
        It is also easy to show that for any $a \in \Gamma(A)$, this acts as a derivation on functions on the manifold $M$, therefore
        $[\Delta, \cdot](a) \in \Gamma(TM)$.    
    \end{proof}
}\label{prop: symbol ses}
% 
\definition{
    The \textbf{k-jet bundle} of a vector bundle $A \rightarrow M$ is
    \begin{equation}
        J^k A := \bigcup J^k_x A,
    \end{equation}
    where the equivalence class is on sections of the vector bundle i.e $J^k_x A = \left\{ [s]^k_x := j^k s(x) \: \big \vert s \in \Gamma(A)\right\}$.
    The induced jet maps $j^k: \Gamma(A) \rightarrow \Gamma(J^k A)$ are defined analogously.
    As in \cref{def:jet bundle for manifold}, we have the long exact sequence
    \begin{equation}
        \begin{tikzcd}
            \ldots \xrightarrow{\pi^{k+1}} J^k A \xrightarrow{\pi^k} J^{k-1} A \rightarrow
            \ldots \xrightarrow{\pi^1} \Gamma(A) \rightarrow 0,
        \end{tikzcd}
    \end{equation}
}\label{def:jet bundles of vector bundles}
% 
\proposition{
    \begin{equation}
        \mathrm{Diff}_k(A,B) \cong \Gamma( (J^k A)^* \otimes B)
    \end{equation}
}\label{prop: differential operators equivalent to jets on vector bundles}

In particular, let us think about differential operators of degree less than $1$ between vector bundles $A,B \rightarrow M$.
\proposition{
    A first order differential operator $\Delta \in \mathrm{Diff}_1(A,B)$ between sections of vector bundles $A,B \rightarrow M$ is uniquely determined by a pair of maps $(\Delta, \delta)$ where
    \begin{align*}
        \Delta : \Gamma(A) &\rightarrow \Gamma(B) \qquad \textrm{is} \:\mathbb{R} \textrm{-linear} \\ 
        \delta : \Gamma(T^*M) &\rightarrow \Gamma(A^* \otimes B) \qquad \textrm{is} \: \mathcal{C}^\infty(M) \textrm{-linear}
    \end{align*}
    such that the Leibniz property holds
    \begin{equation}
        \Delta( f \cdot s) = f \cdot \Delta(s) + \delta(df)(s).
    \end{equation}
    We call this definition of first order differential operator the \textbf{Leibniz Characterisation}.
}\label{prop:Leibniz characterisation}
% 
For example, in the case of 

Considering the differential operator of order 1 in the \cref{eq:symbol ses}, 
\begin{equation}
     0\rightarrow \Gamma(\mathrm{End}(A))\rightarrow \mathrm{Diff}_1(A) \rightarrow \Gamma(TM \otimes \mathrm{End}(A))  \rightarrow 0,
\end{equation}
which is the induced short exact sequence of $\mathcal{C}^\infty(M)$-modules by taking section on 
\begin{equation}
    0\rightarrow \mathrm{End}(A)\rightarrow (J^1 A)^* \otimes A \rightarrow TM \otimes \mathrm{End}(A)  \rightarrow 0,
\end{equation}


% subsection differential_operators_on_vector_bundles (end)
% ------------------------------------------------------------------------------------------
\subsection{Derivations on Vector Bundles} % (fold)
\label{sub:derivations_on_vector_bundles}

% derivations_on_vector_bundles (end)
% section lecture_5_differential_operators_ii (end){}