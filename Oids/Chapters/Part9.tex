% !TeX root = ../main.tex
% 
\newpage
\lhead{\emph{Lecture 9}}
\section{Lecture 9: Dirac Geometry} % (fold)
\label{sec:lecture_9}
Aim of the lecture is to recover Dirac geometry. A lot of the ideas can be traced back to Courant's thesis \citep{Courant1990}. As usual Gualtieri's thesis \citep{Gualtieri2004} serves us well. In the following the underlying field is $\mathbb{R}$.
\subsection{Courant Spaces} % (fold)
\label{sub:courant_spaces}
% 
\definition{
    Let $V$ be a vector space over $\mathbbm{R}$.
    A \textbf{Linear Courant space} is a triple $(C, \langle \cdot, \cdot \rangle, \rho)$ where
    $(C, \langle \cdot,\cdot \rangle)$ is a inner product space (meaning the inner product is bilinear and non-degenerate but \textbf{not} positive-definite) and $\rho: C \rightarrow V$ is a linear homomorphism called the \textit{anchor}.
}\label{def:courant_space}
% 
\definition{
    With respect to the \textit{bilinear form}, a subspace $N \subset C$ and its orthogonal complement $N^\perp := \{ x \in C\; |\; \langle x,y \rangle = 0 \; \forall\, y \in N\}$, need not be disjoint as the inner product is not in general positive definite.
    Therefore a subspace $N \subset C$ is called the following ways if the orthogonality conditions hold:
    \begin{itemize}
        \item $N \subset N^\perp$ is isotropic,
        \item $N \supset N^\perp$ is coisotropic,
        \item $N = N^\perp$ is Lagrangian.
    \end{itemize}
}\label{def:courant_subspaces}
% 
\definition{
    Since $(C, \langle \cdot, \cdot\rangle)$ is an inner product space, we have the usual \textit{musical isomorphisms}:
    \begin{align*}
        C &\xrightarrow{\flat} C^*\\
        C^* &\xrightarrow{\sharp} C
    \end{align*}
}\label{def:musical_iso}
% 
This data of a Courant space implies that we can construct a map $k : V^* \rightarrow V$ such that the following diagram commutes:
% 
\begin{equation}
    \begin{tikzcd}
        C^* \arrow[d,"\sharp"]& V^* \arrow[l,  "\rho^*"] \arrow[d, dashed, "k"]\\
        C \arrow[r, "\rho"]& V
    \end{tikzcd}
\end{equation}
% 
\definition{
    A Courant space is \textbf{exact} when we have the short exact sequence
    \begin{equation}
        0 \rightarrow V^* \xrightarrow{\sharp \rho^*} C \xrightarrow{\rho} V \rightarrow 0.
    \end{equation}
    Define from now on $j = \sharp \rho^*$.
    In such a case, $\mathrm{im}(j) = \mathrm{ker}(\rho) \subset C$ is maximally isotropic (i.e: Lagrangian) in $C$.
    Further, the injective map $j$ is also isotropic in the sense that
    \begin{equation}
        \langle j(V^*), j(V^*) \rangle = 0
    \end{equation}
}\label{def:exact_courant_space}

\definition{
    Given a Courant space $C, C'$, we define
    \begin{itemize}
        \item the \textit{opposite Courant space} $(\overline{C}, - \langle \cdot, \cdot \rangle, \rho)$,
        \item the \textit{direct sum} of two Courant spaces $(C \oplus C, \langle \cdot,\cdot \rangle \oplus \langle \cdot,\cdot \rangle', \rho \oplus \rho')$
    \end{itemize}
}\label{def:construction_courant_space}

\definition{
    For $V \in \mathbf{Vect}_{\mathbb{R}}$ we define the \textbf{Standard Courant space} as $\mathbbm{V} =(V \oplus V', \langle \cdot,\cdot \rangle, \mathrm{pr}_1)$, with bilinear pairing
    \begin{equation}
        \langle v \oplus \alpha, w \oplus \beta \rangle = \frac{1}{2}\left(\alpha(w) + \beta(v)\right)
    \end{equation}
    Note that we can also define a skew-symmetric bilinear form as well, but we will only call the \textit{inner} product the symmetric one. Further note that, the symmetry group preserving orientation is $\mathrm{SO}(d,d)$, the non-compact special orthogonal group. 
}\label{def:std_courant} 
\footnote{
    probably add some example such as B transform here, for posterity
}

\definition{
    A \textbf{Dirac space} D is a Lagrangian subspace of Courant space $(C, \langle \cdot,\cdot \rangle, \rho)$ for which there exists $W \subset V$ and $\overline{W} \subset V^*$ such that the following sequence is exact
    \begin{equation}
        0 \rightarrow \overline{W} \xrightarrow{\sharp \rho^*} D \xrightarrow{\rho} W \rightarrow 0
    \end{equation}
    The space $W = \rho(D) = \faktor{D}{\overline{W}}$ is generally called the \textit{range} of $D$.
    We remark that the Lagrangian condition imposed on the space implies that the bilinear form is $0$ along this subspace.
}\label{def:Dirac space}

\proposition{
    A Dirac space $D \subset (C, \langle \cdot,\cdot \rangle, V)$ specifies a $2$-form on its range $\rho(D)$,
    \begin{equation}
        \omega_D \in \bigwedge^2 W^*
    \end{equation}
    Such that for $w_i = \rho(d_i) + j(\epsilon_i) := \rho(a_i)$, with $d_i \in D$, $\epsilon_i \in V \setminus W$ and some $a_i \in C$ (that restricts to $\omega_i$ in D); the 2-form is:
    \begin{align}
        \omega_D(w_1, w_2) &= \langle a_1, \sharp \rho^*(\epsilon_2) \rangle 
        \label{eq: Dirac 2-form}
        \\
        & = - \langle \sharp \rho^*(\epsilon_1), a_2 \rangle \nonumber
    \end{align}

    \begin{proof}
        D is maximally isotropic so $\forall d_1, d_2 \in D \subset C$,
        \begin{equation}
            \langle d_1, d_2 \rangle = 0.
        \end{equation}
        Since $\rho$ is surjective there exists $w_1, w_2 \in W$ such that $\rho(d_i) = w_i$.
        Consider the extension of elements $w_1, w_2 \in W \subset V$ by $\rho$, that is $a_1, a_2 \in C$ such that
        \begin{equation}
            w_{i} = \rho(a_{i}).
        \end{equation}
        where $a_i = d_i + \sharp \rho^* (\epsilon_i)$
        for some $\epsilon_1, \epsilon_2 \in V^* \setminus W^*$.
% 
        \begin{equation}
            \langle a_1 - \sharp \rho^* (\epsilon_1)\; ,\;  a_2 - \sharp \rho^* (\epsilon_2) \rangle=0
        \end{equation}
        The cross terms must vanish while the non-cross terms carry the non-zero part of the inner product, therefore
        \begin{equation}
            \langle a_1, \sharp \rho^* (\epsilon_2) \rangle = -\langle \sharp \rho^*(\epsilon_1), a_2 \rangle.
        \end{equation}
        sketchy af proof. coset construction? do this
        \begin{tcolorbox}
            think about using construction from vincenzo's notes on branes or from \citep{Marcut}
        \end{tcolorbox}
    \end{proof}
}\label{prop:2_form_on_dirac_space}
% 
\definition{
    An \textit{isotropic} relation (or Lagrangian relation) $\Lambda: A \dashrightarrow B$ between two exact Courant spaces $(A, \langle , \rangle_A, \alpha : A\rightarrow V)$, $(B, \langle , \rangle_B, \beta : B \rightarrow W)$ is a morphism such that $\Lambda \subset A \oplus \overline{B}$ is a Lagrangian subspace with the relations
    \begin{align*}
        a_1 \sim_\Lambda b_1, &\quad a_2 \sim_\Lambda b_2 \\
        \Rightarrow& \quad \langle a_1, a_2 \rangle_A = \langle b_1, b_2 \rangle_B
    \end{align*}
    In this case, we say that elements $a,b \in \Lambda$ are $\Lambda$-related. 
}\label{def:Isotropy_relation}
% 
\definition{
    An isotropic relation $\Gamma:  A\dashrightarrow B $ is a \textbf{Courant morphism} if there exists a map $\gamma: V \rightarrow W$ such that elements of $\mathrm{graph}(\Gamma) \subset B \oplus \overline{A}$ have
    \begin{align*}
        b \oplus a& \in \Gamma \\
        \Rightarrow \beta(b) &= (\gamma \circ \alpha)(a)
    \end{align*}
}\label{def:Courant morphism}
% 
\remark{
    A Courant morphism induces a Dirac space $\Gamma_\gamma \subset B \oplus \overline{A}$ and enters the following short exact sequence,
    \begin{equation}
         0\rightarrow \mathrm{graph}(\gamma^*) \rightarrow \Gamma_\gamma \xrightarrow{\beta \oplus \alpha} \mathrm{graph}(\gamma)  \rightarrow 0
    \end{equation}
    where $\mathrm{graph}(\gamma) \subset W \oplus V^*$ and similarly for $\gamma^*$. This exact sequence is easily read as both piece of the graphs are exact Courant morphisms, on which we take the direct sum.
}\label{rem:Courant morphism induces Dirac}
% 
Courant morphisms are morphism in the category of Courant algebroids, and we have seen that elements are mapped appropriately. A lesser constraint would be to consider maps such that the inner product is preserved.
% 
\definition{
    A \textbf{Courant map} $\Psi: A \rightarrow B$ of Courant algebroids is a linear map together with $\psi: V \rightarrow W$ such that the inner product is preserved,
    \begin{equation}
        \Psi^* \langle \cdot, \cdot \rangle_B = \langle \cdot, \cdot \rangle_A
    \end{equation} 
    and the following diagram commutes
    \begin{equation}
        \begin{tikzcd}
            A \arrow[d,"\alpha"]  \arrow["\Psi"]{r}& B  \arrow["\beta"]{d}\\
            V \arrow[r, "\psi"]& W.
        \end{tikzcd}
    \end{equation}
}\label{def:Courant map}

\proposition{
    A map $\Psi: A \rightarrow B$ is Courant if and only if $\mathrm{graph}(\Psi) \in B \oplus \overline{A}$ is a Courant morphism.
}\label{prop:equivalence of courant maps and morphism}
\begin{proof}
    do later, but similar to \cref{prop:coisotropic_relation}, done in p47 of \citep{Zapata-Carratala2019}.
\end{proof}

\definition{
    The category of exact Courant spaces $\mathbf{Crnt}$ is the category whose objects are exact Courant spaces $C \rightarrow V$ and morphisms are Courant morphisms defined in \cref{def:Courant morphism} that extend linear maps between vector spaces.
}\label{def:Courant category}

\remark{
    It's easy to see that a linear map $\Psi: C \rightarrow D$ is Courant for some linear map $\psi : V \rightarrow W$ (\cref{def:Courant map}) if and only if $\mathrm{grph}(\psi) \subset D \oplus \overline{C}$.
}\label{rem:linear maps as Courant maps}

\remark{
    Since Courant morphisms $\Gamma: A \dashrightarrow B$ between exact Courant spaces $(A \rightarrow V), (B \rightarrow W)$ define a Dirac space $\Gamma_\gamma \subset B \oplus \overline{A}$ (\cref{rem:Courant morphism induces Dirac}), and Dirac spaces specify a 2-form on the range of the anchor, we have a the fibration:
    \begin{equation}
        \Lambda^2 V^* \rightarrow \mathrm{Crnt}(A,B) \rightarrow \mathrm{Vect}(V,W).
    \end{equation}
    Furthermore, by considering Courant automorphisms, we abelianise this fibration to make the following short exact sequence exact:
    \begin{equation}
        0 \rightarrow (\Lambda^2 V^*, +) \rightarrow \mathrm{Aut}_\mathrm{Crnt}(A) \rightarrow \mathrm{GL}(V) \rightarrow 0.
    \end{equation}
}\label{rem:fibration}
% 
It's probably a good idea to stop and think about what this means. By considering Courant algebroids over a vector space, we have extended the available symmetries by 2-forms on said vector space.
% 
\definition{
    An \textbf{isotropic splitting} of an exact Courant space is a map $\nabla$ such that the short exact sequence splits:
    \begin{equation*}
        \begin{tikzcd}
            0 \arrow[]{r}& V^* \arrow["j"]{r}& C \arrow["\rho"]{r}& V \arrow[]{r} \arrow[bend left = 45, color = red, "\nabla", line width=0.25mm]{l} & 0,
        \end{tikzcd}
    \end{equation*}
    with $ \rho \circ \nabla = \mathrm{id}_V$, and the distribution $\nabla(V)$ is isotropic in C with respect to the bilinear form $\langle \cdot, \cdot \rangle$.
}\label{def:Isotropic splitting}
% 
\proposition{
    Exact Courant Spaces always admit an isotropic split. Furthermore, given $\nabla,\nabla' : V \rightarrow C$ two isotropic splits, there exists a 2-form $B \in \Lambda^2 V^*$ such that
    \begin{equation}
        \nabla - \nabla' = j \circ B^\flat
    \end{equation}
    The following proof follows \citep{Zapata-Carratala2019}.
}\label{prop:uniqueness of isotropic splitting up to 2-form}
% 
\begin{proof}
    Consider two not necessarily isotropic sections $\nabla, \nabla' : V \rightarrow C$ such that $\rho \circ \nabla = \mathrm{id}_V = \rho \circ \nabla'$, we can construct the map $\nabla - \nabla' = \gamma$ such that
    \begin{align*}
        \nabla - \nabla' &: V \rightarrow \mathrm{ker}(\rho)
    \end{align*} 
    Since the sequence is exact, then $\gamma(v) \in \mathrm{im}(j)$, so there must  $B \in V^* \otimes V^*$ such that $B^\flat : V \rightarrow V^*$ and
    \begin{equation}
        \gamma = j \circ B^\flat.
    \end{equation}
    In summary we have shown that general sections differ by elements in the Lagrangian subspace $\mathrm{im}(j) = \mathrm{ker}(\rho)$.
    \begin{equation*}
        \begin{tikzcd}
            0 \arrow[]{r}& V^* \arrow["j"]{r}& C \arrow["\rho"]{r}& V \arrow[]{r} \arrow[bend left = 45, color = red, "\gamma", thick]{l} \arrow[bend right = 50, "B^\flat", thick]{ll} & 0.
        \end{tikzcd}
    \end{equation*}
    Now fix a non-isotropic section $\nabla_0$, we show that we can define an isotropic section $\nabla$, we show that defining implicitly
    \begin{equation}
        B_{\nabla_0}(v,w) = -\frac{1}{2}\langle \nabla_0(v), \nabla_0(w) \rangle
    \end{equation}
    allows us to make $\nabla = \nabla_0 + j \circ B^\flat$ into an isotropic map, in the sense
    \begin{equation}
        \langle   \nabla(v), \nabla(w)\rangle = 0.
    \end{equation}
\end{proof}
% 
We note that just like any split short exact sequence, a choice of isotropic section $\nabla$ induces a Courant isomorphism with the standard Courant space,
\begin{equation}
    (C,\langle \cdot,\cdot \rangle, V) \:\overset{\nabla}{\cong} \:(V \oplus V', \langle \cdot,\cdot \rangle, \mathrm{pr}_1).
\end{equation}

EXPAND BELOW, do diagram etc...
Further to \cref{rem:fibration}, the category of exact Courant spaces is explicitly given by $\mathbf{Crnt}(\mathbbm{V,W}) \cong \Lambda^2 V^* \times \mathbf{Vect}(V)$, where morphisms
\begin{align*}
    (B, \phi) : \mathbbm{V} \dashrightarrow \mathbbm{W}
\end{align*}
are both given by an isotropic relation (as defined in \cref{def:Isotropy_relation}) and related by a 2-form.
So
\begin{equation}
    v \oplus \alpha\:  \sim_\psi^B \: w \oplus \beta \qquad \Leftrightarrow \qquad
    \psi(v) = w \quad \mathrm{ and } \quad \alpha + \iota_v B = \psi^* \beta.
\end{equation}
Composition of morphism are given just like a semi-direct product:
\begin{equation}
    (B', \varphi) \circ (B, \psi) = (B' + \psi^* B \:,\: \varphi \circ \psi).
\end{equation}
Finally, by considering automorphisms of Courant spaces, the above argument reduces to
\begin{equation}
    \mathrm{Aut}_{\mathbf{Crnt}}(\mathbbm{V}) = \Lambda^2 V^* \rtimes \mathrm{GL}(V).
\end{equation}
% subsection courant_spaces (end)
% ------------------------------------------------------------------------------------------
\subsection{Dorfman Algebras} % (fold)
\label{sub:dorfman_algebras}
\definition{
    A \textbf{Leibniz algebra} $(\mathfrak{a}, [\cdot,\cdot])$ over $\mathbbm{R}$ is a module with an $\mathbbm{R}$-linear bracket that is \textbf{NOT} skew-symmetric, such that the bracket is a derivation.
    \begin{equation}
        [a, [b,c]] = \left[ [a,b], c\right] + \left[ b,[a,c] \right]
    \end{equation}
    for $a,b,c \in \mathfrak{a}$. 
    Of importance is to note that because the bracket is not antisymmetric, the right/left adjoint maps are not related, i.e:
    \begin{equation}
        \mathrm{ad}^L_a =[a,\cdot] \neq [\cdot, a] = \mathrm{ad}^R_a 
    \end{equation}
}\label{def:Liebniz algebra}
In particular, Lie algebras are Leibniz algebras over $\mathbbm{R}$ with skew-symmetric brackets.
% 
\definition{
    A map $\Psi: \mathfrak{a} \rightarrow \mathfrak{b}$ is a Leibniz morphism if
    \begin{equation}
        \Psi([a,b]_{\mathfrak{a}}) = [\Psi(a), \Psi(b)]_\mathfrak{b}.
    \end{equation}
}\label{def:Liebniz morphism}
% 
\definition{
    Let $\mathfrak{a}$ be a Leibniz algebra and $\mathfrak{h}$ a vector space. A \textbf{Leibniz representation} of $\mathfrak{a}$ over $\mathfrak{h}$ is a pair of maps (or equivalently right and left $\mathfrak{a}$-modules)
    \begin{align*}
        [\cdot,\cdot]_L &: \mathfrak{a} \times \mathfrak{h} \rightarrow \mathfrak{h}\\
        [\cdot,\cdot]_R &: \mathfrak{h} \times \mathfrak{a} \rightarrow \mathfrak{h}
    \end{align*}
    such that all possible combination of Leibniz identities hold.
}\label{def:Liebniz representation}
% 
\definition{
    Given a Leibniz algebra $\mathfrak{a}$ and an Leibniz $\mathfrak{h}$-representation, we define the Leibniz cochain complex $ C^k(\mathfrak{a,h}):= \bigotimes^k \mathfrak{a}^* \otimes \mathfrak{h}$
    \begin{align*}
        \delta^k : C^k(\mathfrak{a,h})& \rightarrow C^{k+1}(\mathfrak{a,h})\\
        h &\mapsto [\cdot, h]_L - [h, \cdot]_R
    \end{align*}
}\label{def:Leibniz cochain complex}
% 
We check explicitly that this exact forms are indeed closed in with this differential.
Let $a,b \in \mathfrak{a}, h \in \mathfrak{h}$ then
\begin{align*}
    \delta^0 h(a) &= [a,h]_L - [h,a]_R \\
    \Rightarrow \qquad \delta^1 \circ \delta^0h (a,b)& = [a, [b, h]_L]_L - [[b,h]_L , a]_R
    - [b, [h,a]_R]_L + [ [h,a]_R, b]_R - \left( [ [a,b],h]_L - [h, [a,b]]_R \right)
    \\
    \qquad &= 0 \qquad \textrm{ using all the Leibniz identities.}
\end{align*}
We extend the differential maps to higher powers and construct a cohomology.
% 
\definition{
    Given a Leibniz algebra $\mathfrak{a}$ and an Leibniz $\mathfrak{h}$-representation, we denote $\mathrm{H}^\bullet(\mathfrak{a,h})$ the Leibniz cohomology of the cochain complex $ (C^k(\mathfrak{a,h}), \delta)$, with differential maps $\delta^2 = 0$.
}\label{def:Leibniz cohomology}
% 
\definition{
    A \textbf{Dorfman Algebra} (or Courant algebra) $(\mathfrak{a}, [\cdot,\cdot])$ over a Lie algebra $\mathfrak{g}$ is a Leibniz algebra $\mathfrak{a}$, together with a morphism of Leibniz algebra
    \begin{equation}
        \rho: (\mathfrak{a}, [\cdot,\cdot]) \rightarrow (\mathfrak{g},[\cdot,\cdot]).
    \end{equation}
}\label{def:Dorfman algebra}
% 
Since $\rho$ is a homomorphism, $\mathrm{ker}(\rho)$ is a 2-sided Leibniz ideal and $\rho(\mathfrak{a}) \subset \mathfrak{g}$ is a Lie subalgebra.
It's probably good to underline that the bracket over the Lie algebra is antisymmetric.
% 
\definition{
    A Dorfman algebra $(\mathfrak{a}, [\cdot,\cdot])$ over a Lie algebra $\mathfrak{g}$ is called \textit{exact} if 
    \begin{equation}
         0\rightarrow \mathfrak{h} \rightarrow \mathfrak{a} \xrightarrow{\rho} \mathfrak{g} \rightarrow 0
    \end{equation}
    with $\mathrm{ker}(\rho) = \mathfrak{h}$ \textbf{and} $\mathfrak{h}$ is an abelian subalgebra of $\mathfrak{a}$, so
    \begin{equation}
        [\mathfrak{h}, \mathfrak{h}] = 0.
    \end{equation}
}\label{def:Exact Dorfman algebra}
We can always build such a sequence but the extra data comes from the fact that $\mathfrak{h}$ is abelian.
% 
\example{
    Consider a right $\mathfrak{g}$-module R over $\mathfrak{h}$ when $\mathfrak{g}$ is Lie
    \begin{equation}
        R : \mathfrak{g} \times \mathfrak{h} \rightarrow \mathfrak{h}.
    \end{equation}  
    Denote the \textit{hemisemidirect product} \footnotemark
    between a Lie algebra and a representation space by $\mathfrak{a} = \mathfrak{h} \oplus \mathfrak{g}$ with bracket
    \begin{equation}
        [h_1 \oplus g_1, h_2 \oplus g_2]_\mathfrak{a} := R(g_1, h_2) \oplus [g_1, g_2]_\mathfrak{g}
    \end{equation}
    for $g_1,g_2 \in \mathfrak{g}$ and $h_1, h_2 \in \mathfrak{h}$.
    We claim that $\mathfrak{a}$ is a Leibniz (in particular Dorfman) algebra if and only if R is a Leibniz representation.
}
\footnotetext{
    Hemi: because we take only the right representation and discard the left.
}
\begin{proof}
    $\bullet$ Suppose $\mathfrak{a}$ is a Leibniz algebra, then the bracket over $\mathfrak{a}$ is a derivation, so by considering the bracket between $a_1,a_2,a_3 \in \mathfrak{a}$,
    \begin{align*}
        [a_1, [a_2,a_3]] = R(g_1, R(g_2,h_3)) \oplus [g_1, [g_2,g_3]].
    \end{align*}
    We use the Leibniz identity for the Lie algebra part and we expand the LHS that also follows Leibniz and match with the $\mathfrak{h}$ part of the RHS.
    Therefore
    \begin{equation}
        R([g_1,g_2], h_3) + R(g_2, R(g_1, h_3)) = R(g_1, R(g_2,h_3))
    \end{equation}
    is a \textit{right} Leibniz representation of $\mathfrak{g}$ on $\mathfrak{h}$.\\
    $\bullet$ Conversely, if R is a Leibniz representation, we can split the brackets and set the left one to zero?
\end{proof}
% 
\definition{
    A Dorfman morphism $ \Psi: \mathfrak{a} \rightarrow \mathfrak{a'}$ is a Leibniz morphism covering the map of algebras $\psi: \mathfrak{g} \rightarrow \mathfrak{g}$ such that
    \begin{equation}
        \rho' \circ \Psi = \psi \circ \rho.
    \end{equation}
}\label{def:Dorfman morphism}
% 
\proposition{
    Let $\mathfrak{a}$ be an exact Dorfman algebra over $\mathfrak{g}$ with anchor $\rho$ and abelian subalgebra $\mathfrak{h} = \mathrm{ker}(\rho)$, then $\mathfrak{a}$ carries a Leibniz representation on $\mathfrak{h}$.
}\label{prop:Exact dorfman implies Leibniz rep}
\begin{proof}
    DO LATER
\end{proof}
% 
If we split the sequence of an exact Dorfman algebra 
\definition{
    Let $\nabla: \mathfrak{g} \rightarrow \mathfrak{a}$ be a section of the following Dorfman algebra with $\mathfrak{g}$-representation
    \begin{equation*}
        \begin{tikzcd}
            0 \arrow[]{r}& \mathfrak{h} \arrow[]{r}& \mathfrak{a} \arrow["\rho"]{r}& \mathfrak{g} \arrow[]{r} \arrow[bend left = 45, color = red, "\nabla", thick]{l} & 0,
        \end{tikzcd}
    \end{equation*}
    with $[\mathfrak{h,h}] = 0$.
    Moreover, sections of the Dorfman algebra provide a lift of the right/left Leibniz representation as
    \begin{align*}
        [\cdot,\cdot ]_L &: \mathfrak{a} \times \mathfrak{g} \rightarrow \mathfrak{g}\\
        [a,g]_L &= [a, \nabla(g)]_\mathfrak{a} \\
        [\cdot,\cdot]_R &: \mathfrak{g} \times \mathfrak{a} \rightarrow \mathfrak{g}\\
        [g,a]_R &= [\nabla(g), a]_\mathfrak{a}
    \end{align*}
    for $a \in \mathfrak{a}, g \in \mathfrak{g}$.
    The \textbf{Leibniz Curvature} of $\nabla$ is
    \begin{align}
        C_\nabla &\in C^2(\mathfrak{g,h}) \nonumber\\
        C_\nabla(g,g') &:= [\nabla(g), \nabla(g')]_\mathfrak{a} - \nabla([g,g']_\mathfrak{g}).
    \end{align}
    We note that $C_\nabla \in \mathrm{ker}(\rho)$ so it takes values in the representation space $\mathfrak{h}$.
    Important to note that $C_\nabla(g,g') \neq C_\nabla(g',g)$.
}\label{def:Leibniz curvature}
% 
\proposition{
    The Leibniz curvature defined above is a cocycle under the differential of Leibniz cohomology defined in \cref{def:Leibniz cohomology},
    \begin{equation}
        \delta C_\nabla = 0.
    \end{equation}
}\label{prop:Leibniz curvature is cocycle}
\begin{proof}
    For $a,b,c \in \mathfrak{g}$,
    \begin{align*}
        \delta C_\nabla(a,b,c) &= [a, C_\nabla(b,c)]_L - [C_\nabla(b,a), c]_R + [C_\nabla(c,a),b]_R \\
        & \qquad + C_\nabla( [a,b]_\mathfrak{g}, c ) - C_\nabla([a,c]_\mathfrak{g},b)
        + C_\nabla(a, [b,c])\\
    \end{align*}
    We then use the section $\nabla$ to lift right and left representations on the Dorfman algebra to find
    \begin{align*}
        \delta C_\nabla(a,b,c) &= \left[\nabla(a) , [\nabla(b), \nabla(c)] \right] - 
        \left[\nabla(a), \nabla([b,c])\right]\\
        & \qquad - \left[ [\nabla(b), \nabla(a)], \nabla(c)\right] +
        \left[\nabla([b,a]), \nabla(c)\right]\\
        & \qquad +\left[[\nabla(c), \nabla(a)], \nabla(b)\right] 
        - \left[\nabla([c,a]), \nabla(b)\right]\\
        & \qquad + \left[\nabla([a,b]), \nabla(c)\right]
        - \nabla\left( \left[ [a,b], c\right]\right)\\
        & \qquad - \left[ \nabla([a,c]), \nabla(b)\right]
        + \nabla\left( \left[ [a,c],b\right]\right)\\
        & \qquad + C_\nabla(a,[b,c])\\
        &=0 
    \end{align*}
    by collecting some terms, using Leibniz everywhere and identifying terms with the last line.
\end{proof}
% 
Since the Leibniz curvature is a cocycle in $C^2(\mathfrak{g,h})$, and the Leibniz differential is closed, we define:
\definition{
    The \textit{Characteristic class} of an exact Dorfman algebra is the 2nd cohomology class $[C_\nabla] \in H^2(\mathfrak{a,g})$.
    This is invariant under different splits since upon choosing a different $\nabla' = \nabla + \eta$ for $\eta : \mathfrak{g} \rightarrow \mathfrak{a}$, a short computation shows
    \begin{equation}
        C_{\nabla'} = C_\nabla + \delta \eta.
    \end{equation}
    Therefore an exact Dorfman algebra carries a Characteristic class 
    \begin{equation}
        [C_\nabla] = [C_{\nabla'}] \in H^2(\mathfrak{a,g}).
    \end{equation}
}\label{def:Characteristic class of Dorfman algebra}
% 
\proposition{
    Two exact Dorfman algebras are isomorphic if and only if their Characteristic classes agree.
    Proof in \citep{Zapata-Carratala2019}, prop 2.5.7
}\label{prop:Isomorphism of Dorfam Algebras}
% 
\example{
    Construct a Dorfman algebra from the Leibniz representation of a Lie algebra $\mathfrak{g}$ over a vector space $\mathfrak{h}$ in the following way. 
    Let $\rho : \mathfrak{g} \rightarrow \mathfrak{h}$ be a Leibniz representation with 
    \begin{align*}
        [\cdot,\cdot]_L : \mathfrak{g} \times \mathfrak{h} &\rightarrow \mathfrak{h}\\
        [\cdot,\cdot]_R : \mathfrak{h} \times \mathfrak{g} &\rightarrow \mathfrak{h},
    \end{align*}
    that can be alternatively be understood as the Dorfman algebra $\mathfrak{w} = \mathfrak{h} \oplus \mathfrak{g}$ with $\rho = \mathrm{pr}_2$.
    Then the semidirect product Dorfman algebra is given by the bracket on $\mathfrak{w}$ given by
    \begin{equation}
        [h \oplus g, h' \oplus g'] = \left([g, h']_L - [h,g']_R + \eta(h,h')\right)
        \oplus [g,g']_\mathfrak{g}
    \end{equation}
    for some bilinear form $\eta: \bigotimes^2 \mathfrak{h} \rightarrow \mathfrak{h}$.
}\label{ex:semidirect product construction of dorfman algebras}
% subsection dorfman_algebras (end)
\subsection{Courant Algebroids} % (fold)
\label{sub:courant_algebroids}
To study of Courant algebroids is to look at a structures encompassing vector bundles (\cref{sub:vector_bundles}), Courant spaces (\cref{sub:courant_spaces}) and Dorfman algebras (\cref{sub:dorfman_algebras}).
Alternatively, one can start from a Lie algebroid and build a Leibniz representation on it, giving a Dorfman structure on it using a semidirect product as in \cref{ex:semidirect product construction of dorfman algebras}.
\example{
    Consider a Lie algebroid $(A, \rho, [\cdot,\cdot])$. Sections of this Lie algebroid enjoy the usual rules of Cartan calculus with operators $\iota, \mathcal{L}, d$ of degrees $-1,0,1$.
    We show that there exists a Leibniz representation of $\Gamma(A)$ on its dual $\Gamma(A^*)$ with the brackets
    \begin{align*}
        [\cdot,\cdot]:_L \Gamma(A) \times \Gamma(A^*) &\rightarrow \Gamma(A^*)\\
        (a, \alpha) &\mapsto \mathcal{L}_a \alpha\\
        [\cdot,\cdot]:_R \Gamma(A^*) \times \Gamma(A) &\rightarrow \Gamma(A^*)\\
        (\alpha, a) &\mapsto \iota_a d \alpha.
    \end{align*}
    To check that this is indeed a Leibniz representation, we need to check all the Jacobi identities. 
    For $a,b \in \Gamma(A)$, $\alpha, \beta \in \Gamma(A^*)$, using \cref{fact:Cartan Calculus}
    \begin{align*}
        [\alpha, [a,b] ]_R &= \iota_{[a,b]} d \alpha\\
        &= \mathcal{L}_{[a,b]} \alpha - d \circ \iota_{[a,b]} \alpha\\
        &= \iota_b \circ d \circ\iota_a \circ d \alpha + \mathcal{L}_a (\iota_b d \alpha)\\
        &= [ [\alpha,a]_R, b]_R + [a, [\alpha, b]_R]_L
    \end{align*}
    and so on...
    Since we have found a Leibniz representation $\Gamma(A) \rightarrow \Gamma(A^*)$, we construct the semidirect product $\Gamma(A) \oplus \Gamma(A^*)$, that is a Dorfman algebra with bracket
    \begin{equation}
        [a \oplus \alpha, b \oplus \beta] = [a,b] \oplus \left( \mathcal{L}_a \beta - \iota_b d \alpha\right).
    \end{equation}
    We note that the lack of skew-symmetry is controlled by 
    \begin{align*}
        [a \oplus \alpha, a \oplus \alpha] &= \mathcal{L}_a \alpha - \iota_a d \alpha\\
        &= d \circ \iota_a \alpha
    \end{align*}
    % \begin{align*}
    %     [ [a,\alpha]_L, b ]_R &= \iota_b d (\mathcal{L}_a \alpha)\\
    %     &= \mathcal{L}_b \mathcal{L}_a \alpha - d (\iota_b \mathcal{L}_a \alpha)\\
    %     &=\\
    % \end{align*}
}\label{ex:Courant algebroid-first example}
% 
\definition{
    A \textbf{Courant algebroid} is the tuple $(E \rightarrow M, \langle \cdot,\cdot \rangle, \rho, [\cdot,\cdot])$ such that $E \rightarrow M$ is a vector bundle, $\langle \cdot,\cdot \rangle$ is a symmetric non-degenerate bilinear form on the fibres. We also ask that the anchor map $\rho: E \rightarrow TM$ is $\mathbbm{R}$-bilinear on sections of the algebroid such that the pushforward of the anchor
    \begin{equation}
        \rho_*: \left( \Gamma(E), [\cdot,\cdot]_E\right) \rightarrow \left(\Gamma(TM), [\cdot,\cdot]\right)
    \end{equation}
    forms a Dorfman algebra.
    Asking for a Dorfman structure means that the bracket on sections of the algebroid is non skew-symmetric but still follows Leibniz identity. 
    So for sections $a,b \in \Gamma(E)$ and $f \in C^\infty(M)$, 
    \begin{align}
        [a, f \cdot b]_E &= f \cdot [a,b]_E + \rho_*(a)(f) 
    \end{align}
    that adjoint action of $a$ on $f$ descends to the pushforward.
    Moreover, we ask for compatibility with the bilinear form in the following way
    \begin{equation}
        \rho_*(a)(\langle b,c \rangle) = \langle [a,b]_E, c \rangle + \langle b, [a,c]_E \rangle.
    \end{equation}
    Finally, we ask for a so called "Tame symmetry", that is failure of the bracket to close is controlled by
    \begin{equation}
        [a,a]_E = D \langle a,a \rangle,
    \end{equation}
    where $D = \sharp \circ \rho^* \circ d : C^\infty(M) \rightarrow \Gamma(E)$.
}\label{def:Courant algebroid}
% 

This definition implies that Courant algebroids are bundles of linear \textit{Courant spaces} (see \cref{def:courant_space})
\begin{equation}
    \rho_x : (E_x, \langle \cdot,\cdot \rangle_x) \rightarrow T_xM,
\end{equation}
whose sections carry a Dorfman algebra structure.

% 
\definition{
    A Courant algebroid $(E \rightarrow M, \langle \cdot,\cdot \rangle, \rho, [\cdot,\cdot])$ is \textit{exact} when it is fibrewise exact.
    Considering the entire bundle of sections, it is then Dorfman exact.
    So we have the short exact sequence of Leibniz algebras (\cref{def:Exact Dorfman algebra})
    \begin{equation}
         0\rightarrow \Omega^1(M)\xrightarrow{i} \Gamma(E) \xrightarrow{\rho_*} \mathfrak{X}(M) \rightarrow 0,
    \end{equation}
    with 
    \begin{equation}
        [i(\Omega^1(M)), i(\Omega^1(M))]_E = 0.
    \end{equation}
}\label{def:Exact Courant algebroid}
% 
The upshot is that all structure introduced in \cref{sub:courant_spaces} and \cref{sub:dorfman_algebras} applies here. 
In particular we recall that exact Dorfman algebras $\mathfrak{h} \oplus \mathfrak{g}  \rightarrow \mathfrak{g}$ are represented by their characteristic classes $[C_\nabla] \in H^2(\mathfrak{g,h})$.

\definition{
    Let $E \rightarrow M$ be an exact Courant algebroid, such that $\left( \Gamma(E), [\cdot,\cdot]_E\right) \rightarrow \left(\Gamma(TM), [\cdot,\cdot]\right)$ is an exact Dorfman algebra, then this algebra is represented by an element 
    \begin{equation}
        [\eta] \in H^2( \mathfrak{X}(M), \Omega^1(M))
    \end{equation}
    called the \textbf{\v{S}evera class} of $E$.
    Since this cohomology has coefficients in the space of one-forms, it also defines a closed $H$-flux
    \begin{equation}
        [H] \in H^3(M).
    \end{equation}
    The relation being $\quad \delta \eta = 0\: \Leftrightarrow \: d H = 0$.
}\label{def:Severa class}
% 
\example{
    The \textit{Standard Courant algebroid}, found when describing Supergravity is \\ 
    $T \mathbbm{M} = TM \oplus_M T^*M$.
    It has inner Courant bracket and Dorfman bracket given by
    \begin{align*}
        \langle X \oplus \alpha, Y \oplus \beta \rangle &= \frac{1}{2} (\alpha(Y) + \beta(X))\\
        [X \oplus \alpha, Y \oplus \beta] &= [X,Y] \oplus \left( \mathcal{L}_X \beta - \iota_Y d \alpha \: {\color{red} + \iota_X \iota_Y H}\right).
    \end{align*}
    The standard bracket can be twisted by an H-flux. This standard result is made more clear in \citep[section 2.5.2]{Zapata-Carratala2019}.
}\label{ex:Standard Courant algebroid, supergravity}
% 
\definition{
    Let $(E \rightarrow M, \langle \cdot,\cdot \rangle, \rho, [\cdot,\cdot])$ be a Courant algebroid, a \textbf{Dirac Structure} supported on submanifold $Q \subset M$ is a subbundle $D \subset E|_Q$, on the restricted support, such that 
    \begin{itemize}
        \item $D_q$ is maximally isotropic with respect to $\langle \cdot, \cdot \rangle_q$ for all $q \in Q$
        \item $\rho : D \rightarrow TQ$ is an anchor over $Q$
        \item the space of sections on Dirac space $D$ is involutive,
        \begin{equation}
            [ \Gamma(D), \Gamma(D)] \subseteq \Gamma(D).
        \end{equation} 
    \end{itemize}
    In other words, the bundle $D$ is fiberwise a \textit{Dirac space} (see \cref{def:Dirac space}).
}\label{def:Dirac Structure}
% 
Since Dirac structures are fiberwise Dirac, extending \cref{prop:2_form_on_dirac_space} to the entire bundle, there exists a 2-form given by \cref{eq: Dirac 2-form}
\begin{align}
    \omega_q^D &\in \Lambda^2(\mathrm{im}(\rho_q)) \subset \Lambda^2 T_q^* M\\
    d \omega_q^D(X_q,Y_q, Z_q) &= \langle [X, Y], Z \rangle \big|_q
\end{align}
NEEDS more work up here
% subsection courant_algebroids (end)
% section lecture_9 (end)