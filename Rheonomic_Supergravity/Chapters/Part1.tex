% !TeX root = ../main.tex

% --------------------------------------------------------------------------------------------
\section{Lecture 1} % (fold)
\label{sec:lecture_1}
Mostly covered in \citep{Wise2010}, lectured by comrade Andrew Beckett.

\subsection{Klein Geometry} % (fold)
\label{sub:klein_geometry}
In the lecture, let $M$ denote a smooth manifold, $G$ a Lie group and throughout, $G \circlearrowright M$ \textit{transitively}.
\footnotemark
\footnotetext{
    That is $\forall x,y \in M \: ,  \exists g \in G \textrm{ such that } g \cdot x = y$.
}
% 
\definition{
    A homogeneous space $(M,G)$ is a smooth manifold $M$ together with a transitive Lie group action $G \circlearrowright M$.
    Right now we stick with a \textbf{left} group action, but might change in the notes depending on Andrew's taste.
    We study homogeneous spaces by considering subgroups $H \subset G$, and looking at the geometry of the quotient space, which can be thought as the space modelled on $H$.
}\label{def:Homogeneous space}
% 
If we pick a point $o \in M$ to act as an "origin", the \textit{stabiliser} which fixes $o$ is the subgroup 
\begin{equation}
    \mathrm{stab}_o (G) :=  \left\{ g \in G \:|\:  g \cdot o=  o \right\}.
\end{equation}
We denote this subgroup of symmetries leaving $o$ invariant $\mathrm{stab}_o (G) = G_o$, or the \textit{isotropy} subgroup at $o$.
There exists an isomorphism
\begin{align*}
    M &\cong G / G_o \\
    g \cdot o &\mapsto g \: G_0
\end{align*}
which is a  $G_0$-invariant diffeomorphism. It turns out that this is independent of the origin chosen as stabiliser subgroups at different points are conjugate by transitivity of the group action.
% 
\definition{
    Given a homogeneous space $(M,G)$, and closed subgroup $H \subset G$ the space with "features" $H$ is $G/H$ and it is also a homogeneous space.
    Sometimes, this quotient is called the homogeneous space in the literature.
}\label{def:Quotient Homogeneous space}
% 
% 
\definition{ (Klein Geometry)\\
    A (smooth connected) Klein geometry is a pair $(G,H)$ where $H$ closed subgroup of G such that $G/H$ is connected. 
    Note that 
    \begin{equation}
        \begin{tikzcd}
            G \arrow["\pi"]{d} \\ G/H
        \end{tikzcd}
    \end{equation}
    is automatically a principal $H$-bundle, because the fibres are left $H$-cosets and the action of $H$ on the fibres is obviously an $H$-torsor.
    % The last condition is not necessary and not always there but it is convenient.
}\label{def:Klein geometry}
% 
% 
Consider a Klein geometry $(G,H)$, then the Lie functor gives us a pair of Lie algebras 
$(G,H) \xrightarrow{Lie} (\mathfrak{g}, \mathfrak{h})$, where $\mathfrak{h}$ is a Lie subalgebra of $\mathfrak{g}$.
Notably, this induces a short exact sequence of $H$-modules. 
\footnote{
    A vector space on which $H$ acts linearly; a representation of $H$. Here, $H$ acts via the adjoint representation on \(\mathfrak{g}\) and \(\mathfrak{h}\) and via the quotient representation on \(\mathfrak{g}/\mathfrak{h}\).
}
\begin{equation}
    \begin{tikzcd}
        0 \arrow[]{r}& \mathfrak{h} \arrow[]{r} & \mathfrak{g} \arrow[]{r}& \mathfrak{g}/ \mathfrak{h} \arrow[bend right=-60, color= green]{l} \arrow[]{r}& 0
    \end{tikzcd}
    \label{eq: ses of H-modules}
\end{equation}
% 
\definition{
    If a short exact sequence of $H$-modules as above splits, then we call $(\mathfrak{g},\mathfrak{h})$ \textbf{reductive}. Equivalently\footnote{$\mathfrak{m}$ is the image of the splitting map.},
    there exists an $H$-invariant subspace $\mathfrak{m} \subseteq \mathfrak{g}$ such that
    \begin{align}
        \mathfrak{g} = \mathfrak{h} \oplus \mathfrak{m}.
    \end{align}
   Invariance of $\mathfrak{m}$ under the action of $H$ is equivalent to  
    \begin{equation}
        [\mathfrak{h}, \mathfrak{m}] \subseteq \mathfrak{m}. 
    \end{equation}
    As \(H\)-modules, we have \(\mathfrak{m}\cong\mathfrak{g}/\mathfrak{h}\). The reductive case will always be denoted by green throughout the rest of this lecture.
}\label{def:reductive}
% 
\definition{
    A \textit{reductive} pair of Lie algebras $(\mathfrak{g,h})$ is called \textbf{symmetric} if it further follows the condition
    \begin{equation}
        [\mathfrak{m,m}] \subseteq \mathfrak{h}.
    \end{equation}
    We will follow the convention of denoting in blue the symmetric case from now on.
}\label{def:Symmetric pair}
% 
The terminology is summarised in the table below, where $\mathfrak{m}$ is vector space complement to $\mathfrak{h}$ in $\mathfrak{g}$ (it is not $H$-invariant in general):
\begin{center}
$
    \begin{tabu}{ c|c|c|c }  
    \subseteq&General&Reductive& Symmetric \\
    \hline
    \left[\mathfrak{h}, \mathfrak{h}\right]&\mathfrak{h}&{\color{green}\mathfrak{h}}&{\color{blue}\mathfrak{h}}\\
    \left[\mathfrak{h}, \mathfrak{m}\right]&\mathfrak{h} \oplus \mathfrak{m} & {\color{green}\mathfrak{m} } &{\color{blue}\mathfrak{m}}\\
    \left[\mathfrak{m}, \mathfrak{m}\right]& \mathfrak{h} \oplus \mathfrak{m}  &{\color{green} \mathfrak{h} \oplus \mathfrak{m}}& {\color{blue} \mathfrak{h}}    
    \end{tabu}
$
\end{center}
\example{
    Consider the homogeneous space  $G \circlearrowright M$. The left group action induces a morphism of tangent bundles
    \begin{equation}
        (L_g)_* : T_x M \rightarrow T_{g \cdot  x}M.
    \end{equation}
    Considering the isotropy subgroup $H := G_o$ for a chosen origin $o \in M$, then for $h \in H$, the pushforward $(L_h)_* \in \mathrm{GL}(T_oM) $ is an automorphism of the tangent bundle at the origin.
    Further, the map
    \begin{align*}
        \rho : H &\rightarrow \mathrm{GL}(T_oM) \\
        h &\mapsto (L_h)_*
    \end{align*}
    is a representation of $H$ on $T_oM$. We call this the \textit{linear isotropy representation} of $(M, G, o)$.
}\label{ex:Linear isotropy representation}
% 
\remark{
    For a Klein geometry $M \cong G/H$, we have the sequence of isomorphisms of $H$-representations
    \begin{equation}
        T_oM \xrightarrow{\sim} T_H(G/H) \xrightarrow{\sim } \mathfrak{g/h} 
        \quad {\color{green} \cong m}
        \label{eq:isomorphism of H reps}
    \end{equation}
    which can be constructed from the transitivity of the group action.
%  
    However, looking at the tangent bundles on the group manifolds we find the following short exact sequence
    \begin{equation}
         0\rightarrow T_eH \xrightarrow{(L_e)_*} T_eG \xrightarrow{(\pi_e)_*} T_H(G/H) \rightarrow 0,
    \end{equation}
    which is really the same as \cref{eq: ses of H-modules}.
    But noticing the isomorphisms of $H$-representations in \cref{eq:isomorphism of H reps}, the tangent bundle of the manifold $M$ at the origin is modelled on 
    \begin{equation}
        T_oM \cong \mathfrak{g}/ \mathfrak{h} \quad {\color{green} \cong \mathfrak{m}}
    \end{equation}
    This facts leads us to notice a broader correspondence between geometry and algebra for Klein geometries.
}
% 
\proposition{ (Correspondence)\\
    Let $(G,H)$ be a Klein geometry, then linear structures on $\mathfrak{g/h}$ correspond to geometric structures on the manifold $M \cong G/H$ in the following sense:
    \begin{equation}
        \left\{ 
        H\textrm{-invariant tensors of } \mathfrak{g/h} \cong T_oM
         \right\}
         \qquad \longleftrightarrow \qquad
         \left\{ 
         G\textrm{-invariant tensor fields on } M
          \right\}
    \end{equation}
}\label{prop:correspondance between algebraic and geometric}
% 
\begin{proof}
    $(\Rightarrow)$\\
    Consider $\tau \in \bigotimes T_oM$ an $H$-invariant tensor, such that $(L_h)_* \tau = \tau$ for $h \in H$.
    Consider the map $\tau \mapsto T_o$ for a fixed origin, we then left translate by $g \in G$ to span the manifold. 
    So define 
    \begin{align*}
        T_{g \cdot o} = (L_g)_* T_o \:\in \Gamma( \bigotimes T_{g \cdot o}M).
    \end{align*}
    for all $g \in G$.
    This map is well defined since if $g \cdot o = g' \cdot o$ are two left translation yielding the same tensor field at $T_{g \cdot o} = T_{g' \cdot o}$, then $g^{-1}g' \in H$ an isotropy.
    But $\tau = T_o$ is H-invariant implying that $(L_{g^{-1}g'})_* T_o = T_o$ is $G$-invariant.
    \\
    $(\Leftarrow)$\\
    Given a $G$-invariant $T \in \Gamma(TM)$, the evaluation map $\mathrm{ev}_o : \Gamma(TM) \rightarrow T_oM$ sends $T \rightarrow T_o$ which has isotropy group $H$.
\end{proof}
% 
In particular, a pseudo inner product on $\mathfrak{g/h}$ gives rise to a pseudo-riemannian metric on M.
% 
\definition{
    A Metric Klein geometry $(G,H, \eta)$ is 
    \begin{itemize}
        \item $(G,H)$ a Klein geometry
        \item $\eta$ is a pseudo inner product on $\mathfrak{g/h}$ which is $H$-invariant in the sense describe in the proposition above.
    \end{itemize}
}
% 
Let's just recall the isometries of flat spacetime for completeness.
\definition{
    The Poincaré group $\mathrm{ISO}(d-1,1)$ of $d$-dimensional Minkowski spacetime is the isotropy group that leaves the split quadratic form invariant. 
    We also call it the \textit{inhomogeneous special orthonormal} group as it consists of disconnected components.
    It consists of the semidirect product of the Lorentz group (not \textbf{necessarily} orthochronous) and the group of spacetime translations,
    \begin{equation}
        \mathrm{ISO}(d-1,1) \cong \mathrm{SO}(d-1,1) \ltimes \mathbbm{R}^{d-1,1}
    \end{equation}
    with multiplication
    \begin{equation}
        (g, \alpha) \cdot (f, \beta) = (g f, \alpha + f \cdot \beta)
    \end{equation}
}\label{def:Poincaré group}
% 
\example{
    % 
    Using our new-found understanding of the Poincaré group, we understand it as a homogeneous space. For example, taking $G = ISO(d-1,1)$ and considering the isotropy group at an origin $o$ to be the Lorentz group $H = \mathrm{SO}(d-1,1)$. 
    The manifold $G/H \cong \mathbbm{R}^{d-1,1}$ is connected and at the level of algebras $\mathfrak{g/h} \cong \mathbb{R}^{d-1,1}$.
    Further, the bilinear form on $\mathfrak{g/h}$ is invariant under rotations ($H$-invariant). 
    Therefore, the Poincaré group with its Lorentz subgroups is a Klein geometry. 
    A similar argument can be made for Euclidean Poincaré groups.
    It is interesting to note however, that using the correspondence established in \cref{prop:correspondance between algebraic and geometric}, the Lorentz invariant bilinear form on $\mathfrak{g/h}$ corresponds to Poincaré invariant tensor fields on the full Minkowski spacetime.
}\label{ex:examples of metric klein geometry}
% 
% 
\example{
    We can summarise the ideas presented above in the following reductive diagrams, where arrows represent the subgroup that is quotiented out:
    \begin{equation*}
        \begin{tikzcd}
        &ISO(d-1,1) \arrow["\textrm{spacelike hyperplane}", swap]{dl} 
        \arrow["\mathbb{R}^{d-1,1} = \textrm{ event space}"]{dr}&\\
        ISO(d-1) \arrow["\mathbb{R}^{d-1}"]{dr}
        &&SO(d-1,1) \arrow["\textrm{velocity space}"]{dl} \\
        &SO(n)&
        \end{tikzcd}
    \end{equation*}
    By "event space", we mean that we quotient by all possible translations of spacetime, while the "spacelike hyperplane" is quotienting by the Lorentz boosts. 
    \footnote{
        not sure here
    }
    In $4$ spacetime dimensions, the more familiar diagram:
    \begin{equation*}
        \begin{tikzcd}
            &SO(4,1) \arrow["dS"]{dr} \arrow["H^4"]{dl}&\\
            SO(4) \arrow["S^3"]{dr}&&SO(3,1) \arrow["\textrm{velocity space}"]{dl}\\
            &SO(3)&
        \end{tikzcd}
    \end{equation*}
    I don't know much about dS, so maybe someone can complete here.
}\label{ex:diagrams}
% ------------------------------------------------------------------------------------------
% subsection klein_geometry (end)
% 
\subsection{Cartan geometry} % (fold)
\label{sub:cartan_geometry}
Assuming familiarity with Maurer-Cartan form $\omega_G \in \Omega^1(G, \mathfrak{g})$ for a Lie group $G$, which are the $G$-invariant one forms on $G$, we review the more general notion of connections on a principal $G$-bundle.
For some reason, the literature often prefers \textit{right} action on principal bundles. idk

\definition{
    An Ehresmann connection on a principal right $G$-bundle $(P \rightarrow M)$ is a $\mathfrak{g}$-valued one-form $A$ on $P$, that is a map
    \begin{equation}
        \omega : TP \rightarrow \mathfrak{g}
    \end{equation}
    such that
    \begin{align*}
        (R_g)^* \omega &= \mathrm{Ad}_{g^{-1}} \omega \:\textrm{ for all } g \in G\\
        \omega(\xi_X) &= X \: \textrm{ for } X \in V_p.
    \end{align*}
    Equivalently an Ehresmann connection is a \textit{choice} of Horizontal distribution such that $T_pP = V_p \oplus H_p$ where $H_p$ is $G$-equivariant. This is better explained in the EKC notes from last year.
}\label{def:Ehresmann Connections}
% 
We now turn our attention to the case of a principal bundle modelled on a reduced geometry.
% 
\definition{ (Cartan Geometry)\\
    A Cartan Geometry $(\pi: P \rightarrow M, A)$ modelled on a Klein geometry $(G,H)$ is a principle right H-bundle $P \rightarrow M$ with a \textbf{Cartan connection}  $A \in \Omega^1(P, \mathfrak{g})$ satisfying the conditions:
    % 
    \begin{itemize}
        \item $A_p : T_pP \rightarrow \mathfrak{g}$ is a linear \textbf{isomorphism} (or simply put a $\mathfrak{g}$-valued one-form for on $P$).
        \item $(R_h)^* A = \mathrm{Ad}_{h^{-1}} \circ A$ for all $h \in H$
        \item $A(\xi_X) = X$ for $X \in \mathfrak{h}$ and fundamental vector field $\xi_X$ of $H \circlearrowright P$.
    \end{itemize}
    % 
    Importantly, the Cartan connection one-form takes values in the larger Lie algebra $\mathfrak{g}$, because the tangent space of this principal bundle is isomorphic to the larger Lie algebra $\mathfrak{g}$.
    Morally speaking, the algebra $\mathfrak{g} \cong T_pP$, while $\mathfrak{h} \cong V_p$ and $\mathfrak{m} \cong H_P \cong T_{\pi(p)}M$. 
    % 
}\label{def:Cartan geo}
% 
\remark{
    Consequently, in a Cartan geometry we identify because of the first condition $\mathrm{dim} (P) = \mathrm{dim} (G)$ and $\mathrm{dim} (M) = \mathrm{dim} (G/H)$.
    Moreover, the Cartan connection isomorphisms implies that upon restricting to the subalgebra $\mathfrak{h}$ and choosing an origin
    \begin{align*}
        (A_o)^{-1} :& \mathfrak{h} \rightarrow \mathfrak{X}_{\textrm{vert}}(P)\\
        X &\mapsto \xi_X.
    \end{align*}
}
% 
\remark{
    For a Cartan geometry $(P \rightarrow M: A)$ modelled on a \textit{metric} Klein geometry $(G,H)$, the correspondence \ref{prop:correspondance between algebraic and geometric} implies that the isomorphism
    \begin{equation}
        T_xM \cong T_pP / \mathrm{ker}(\pi_*) \cong \mathfrak{g}/\mathfrak{h} 
        \quad {\color{green}\cong \mathfrak{m}}
    \end{equation}
    give a metric of the same sign to the manifold $M$. Where in the above $x := \pi(p)$.
    In general, the full tangent bundle of the manifold can be thought of as the associated bundle of the $H$-action
    \begin{equation}
        TM \cong P \times_H \mathfrak{g}/\mathfrak{h}
    \end{equation}
}
% 
\definition{
    The curvature of a Cartan Geometry $(\pi: P \rightarrow M, A)$ modelled on a Klein geometry $(G,H)$ is 
    \begin{equation}
        F(A) = dA + \frac{1}{2} [A,A] \in \Omega^2(P, \mathfrak{g})
    \end{equation}
}
We summarise the different connections in a Cartan geometry, whether it is reductive or not in the following diagram.
We denote $e : TP \rightarrow \mathfrak{g/h} \cong TM$ as the veilbein of this geometry,
while the Ehresmann connection $\omega : TP \rightarrow \mathfrak{h}$ is denoted in green.
\begin{equation}
    \begin{tikzcd}
        && {\color{green}\mathfrak{h}}\\
        TP \arrow["A"]{r} \arrow[bend left, color = green, "\omega"]{urr}
        \arrow[bend right, "e"]{drr}
        & 
        \mathfrak{g} \arrow[color = green]{ur} \arrow[->>]{dr}& \\
        && \mathfrak{g/h} \: {\color{green} \cong \mathfrak{m}}
    \end{tikzcd}
\end{equation}
So in the reductive case 
\begin{equation}
    A = \omega + e.
\end{equation}
Considering the curvature of these connections, we have the diagram
\begin{equation}
    \begin{tikzcd}
        && {\color{green}\mathfrak{h}}\\
        \bigwedge^2 (TP) \arrow["F"]{r} \arrow[bend left, color = green, "\hat{F}"]{urr}
        \arrow[bend right, "T"]{drr}
        & 
        \mathfrak{g} \arrow[color = green]{ur} \arrow[->>]{dr}& \\
        && \mathfrak{g/h} \: {\color{green} \cong \mathfrak{m}}
    \end{tikzcd}
\end{equation}
with the understanding that in the reductive case, 
\begin{equation}
    F = \hat{F} + T
\end{equation}
where $T$ is called the \textit{torsion} and $\hat{F}$ is the Ehresmann part of the curvature.
\remark{
    A short calculation shows that if the geometry is reductive then the Cartan curvature
    \begin{align*}
        F[A] &= dA + \frac{1}{2}[A,A]_\mathfrak{g} \\
        &=d \omega + de + \frac{1}{2} [\omega + e, \omega + e]_\mathfrak{g} \\
        &= \left(
        d \omega + \frac{1}{2}[\omega, \omega]_\mathfrak{h} + [\omega , e]_\mathfrak{h} +\frac{1}{2} [e,e]_\mathfrak{h}
        + de + \frac{1}{2}[\omega,\omega]_\mathfrak{m} + [\omega,e]_\mathfrak{m} + \frac{1}{2}[e,e,]_\mathfrak{m}
        \right)
    \end{align*}
    Now because $\omega$ is an Ehresmann connection and it reduces to a Maurer-Cartan form on the fibres we its curvature $2$-form
    \begin{equation*}
        d \omega + \frac{1}{2} [\omega , \omega]_\mathfrak{h} = \Omega(\omega).
    \end{equation*}
    If we are in the reductive case, we recall that $[\mathfrak{h,m}] \subseteq \mathfrak{m}$, so the bracket $[\omega, e]_\mathfrak{h} = 0$.
    With further restriction in the symmetric case, $[\mathfrak{m,m}] \subseteq \mathfrak{h}$, the bracket $[e,e]_\mathfrak{h} = 0$, but we won't assume it in general so instead define a covariant derivative on $\mathfrak{m}$
    \begin{equation}
        d^\omega = d + \omega.
    \end{equation}
    \\
    So in conclusion, the Ehresmann part of the curvature in a Cartan geometry is
    \begin{equation}
        \hat{F} = \Omega(\omega) + \frac{1}{2}[e,e]_\mathfrak{h},
    \end{equation}
    while the torsion is
    \begin{equation}
        T = d^\omega e + \frac{1}{2} [e,e]_m
    \end{equation}
    Some interpretation of flatness and torsion freedom needed here.
}

% subsection cartan_geometry (end)
% section lecture_1 (end)