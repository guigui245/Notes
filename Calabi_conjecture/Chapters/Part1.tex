% !TeX root = ../main.tex
% 
\lhead{\emph{Lecture 1}}
\section{Lecture 1} % (fold)

\subsection{Motivation} % (fold)
\label{ssub:the_endgame}

Complex manifolds provide an interface between algebra and analysis.
The Calabi conjecture has many applications in algebraic geometry, for example the proof that $\mathbb{C}P^2$ is rigid ( holomorphic implies biholomorphic ), or the fact that Fano manifolds are simply connected.
Applications also to moduli spaces, for example the sides of triangles have a structure in their own rights.
The Calabi conjecture has application to finding Riemannian manifolds with vanishing Ricci curvature but not vanishing Riemannian curvature tensor.

How many objects in $\mathbb{R}^n$ satisfy certain conditions? Classify the zeroes of degree d polynomials. This is a hard question to answer. And the answers to such questions are very dependant on the category on which the analysis is done. \\
We will be working in the complex category and study \textit{Complex Algebraic Geometry}.
So for example the question of finding all rational points on $x^2 + y^{2} = 1$ makes sense also on $\mathbb{C}^2$, so one can try to use differential geometry and hope to get something like number theory. This is called Arakelov geometry.
% subsection more_motivation (end)
% 
% 
\subsection{Complex manifolds} % (fold)
\label{sub:complex_manifolds}
We study the loci of polynomials or more generally, analytic function on $\mathbb{C}^{n}$.
% 
\definition{
    A complex manifold $M$ is a smooth manifold that is diffeomorphic to an open subset to $\mathbb{C}^n$, such that transition function are biholomorphic.
}
% 
\definition{
    An holomorphic function $f: U \subset \mathbb{C}^n \rightarrow \mathbb{C}$, is a function $f$ is $\mathrm{C}^1$ and $\frac{\partial f }{\partial \overline{z}^i} = 0$ for $i\in \{ 1, \ldots, n \}$.\\
    But equivalently using Hartogs' theorem, $f$ is $\mathrm{C}^1$ and $\frac{\partial f }{\partial \overline{z}^i} = 0$ (and by extension \textit{separately analytic}), then it is holomorphic.
    Another way to say this is $f$ locally it admits a power series expansion. 
    \textbf{All} of these are false for real smooth function but true for complex function.
}
% 
\remark{
    Holomorphicity in more than $\mathbf{2}$ variables is very different from the one dimension counterpart. An holomorphic function on $\mathbb{C}^n \setminus \{ 0 \} $ can be holomorphic extended to $\mathbb{C}^n$!! This means that singularities of holomorphic function in more than 2 variable \textbf{cannot} be isolated.
    This is called \textit{Hartogs'} phenomenon.
    Moreover, $ \mathbb{B}^1 \times \mathbb{B}^1 $ is \textbf{NOT} biholomorphic to $\mathbb{B}^2$ unlike the real analogue. 
}
\example{
    A first example of a complex manifold is given by the trivial inclusion of open $U \hookrightarrow \mathbb{C}^n$. But it is difficult to come up with \textbf{compact} complex manifold.
}

\definition{
    The holomorphic tangent space $\mathrm{T}^{1,0}_p \mathbb{C}^n$ is the vector space generated by $\frac{\partial}{\partial \overline{z}^i}$, is spanned by derivative of complex functions.
    Similarly, we define anti-holomorphic vector space and their respective duals. 
}
% 
\remark{
The complexification of a real tangent space $\mathrm{T}_p M \otimes \mathbb{C} = \mathrm{T}_p^{1,0}M \oplus \mathrm{T}_p^{0,1}M$ for real even-dimension manifold. 
But here, the complexified vector bundle $T_{\mathbb{C}}M$ is \textbf{NOT} a holomorphic vector bundle, but it is a complex vector bundle over $M$. More on this below.
}
% 
\definition{
    We define holomorphic/anti-holomorphic $(p,q)$-form as the space $\Omega^{(p,q)}(M)$ generated by $dz^i \wedge d \overline{z}^j$.  
    The exterior derivative $d: \Omega^k \rightarrow \Omega^{k+1}$ splits into holomorphic and anti-holomorphic $\partial$ and $\overline{\partial}$ in the obvious way, with $d = \partial + \overline{\partial}$.\\
    e.g:
    $\overline{\partial} (\overline{z^1}d \overline{z}^2) = d \overline{z}^1 \wedge d \overline{z}^2$.
}
\theorem[Poincaré Lemma]{
    Let $U \subset M$ open, contractible subset of real manifold $M$, then for $\alpha \in \Omega^p(U)$, we have $d \alpha = 0 \Rightarrow \alpha = d \beta$.
    In the complex case, the analogous result for $(p,q)$-form holds for open balls in $\mathbb{C}^n$.
}
% 
To build submanifold of $\mathbb{C}^n$, it is useful to prove the complex equivalent of the implicit function theorem (IFT). This is a good way to generate manifold as the zero locus of some function.
% 
\theorem[Complex inverse function theorem]{
     If $f: \mathbb{C}^n \rightarrow \mathbb{C}^n$ is holomorphic and $\mathrm{D}f(0)$ is invertible, then f is a local biholomorphism near $0$.
}\\
This leads to analogous implicit function theorem and constant rank theorem for complex functions.
% 
\definition{
A complex submanifold $S \subset M$ is a complex manifold such that the inclusion map is a \textit{holomorphic} embedding. We show that it is a complex manifold by using the implicit function theorem.
In a neighbourhood of $s \in S$, there is a coordinate chart such that $S$ is the zero locus of the first few coordinate function.
}
% 
\definition{
    A regular value $p \in N$ for function $f: M \rightarrow N$ is such that $df : T_{f^{-1}(p)}M \rightarrow T_p N$ is surjective in a neighbourhood of p. It can be shown that the set of regular values for holomorphic maps is \textit{dense} in $\mathbb{C}^n$.
}

The IFT implies that if $f: M \rightarrow N$ a holomorphic map between complex manifold, and if $p \in N$ a regular value for $f$, then $f^{-1}(p)$ is a complex sub-manifold of $M$ of dimension $m-n$.

\example{
    Non-trivial example of complex manifold
     \begin{itemize}
         \item $x^{2} + y^{2} = 1$ is an example because  complex $\nabla f = (2x, 2y)$ non zero, so this is a non-compact complex sub-manifold of $\mathbb{C}^2$.
         It is a  \textit{fact} that there are no compact complex submanifold of $\mathbb{C}^n$ (unless a point).

         \item More generally, we will prove in prop. \ref{prop_cte} that a holomorphic function on a compact complex manifold is \textbf{constant}.  This statement implies the above since if there was a compact submanifold of $\mathbb{C}^n$, then the embedding is a constant function. So by the implicit function theorem, the coordinates on the submanifold are constants, giving a point.
     \end{itemize}
}
% 
\proposition{
    A holomorphic function on a compact complex manifold is \textit{constant}.
\begin{proof}
    $f : M \rightarrow N$ holomorphic on complex manifold with $f = u + i v$ then if $u$ attains its maximum at $p$, then in any coordinates chart of a complex manifold. But we know that harmonic function satisfy the mean value property (the value at point $p$ is given by the average over a ball centred at that point). So if the maximum is achieved at the centre of the ball then it is constant on that ball. Then using connectedness, we prove the claim.
\end{proof}
}\label{prop_cte}
% 
\example[Examples of compact complex manifolds]{ \noindent
     \begin{itemize}
         \item Use a quotient construction : $\faktor{\mathbb{C}^n}{\Lambda}$, the complex torus for lattice $\Lambda$. 
         \item Hopf surface $\faktor{\mathbb{C}^2\setminus \{ 0 \} }{z \sim 2^n z} \cong S^1 \times S^3$
         \item Chart for $\mathbb{C}P^n$: the homogeneous chart given by $z^i = \frac{X^i}{X^0}$ for $X^0 \neq 0$ with transition map given by ratio of $2$ holomorphic functions.
     \end{itemize}    
}
% subsection complex_manifolds (end)
\subsection{Projective Varieties} % (fold)
\label{sub:projective_varieties}
Let's study other kinds of compact submanifolds of $\mathbb{C}P^n$, which must be a quotient construction by above arguments.
% 
\example{
    Consider the set $\sum a_i X^i = 0$ in $\mathbb{C}^{n+1}$ for $a_i \not\equiv 0$ and taking the quotient map to homogeneous coordinates, we get a submanifold of $\mathbb{C}P^n$ ad in fact it is biholomorphic to $\mathbb{C}P^{n-1}$. 
}
    
More generally, taking a homogeneous polynomial $F: \mathbb{C}P^n \rightarrow \mathbb{C}$, then ker($F$) is a well-defined subset of $\mathbb{C}P^n$ but \textbf{not} necessarily a submanifold.\\
For example $Y^2X = Z^3$ in $\mathbb{C}P^2$ is not a submanifold, as the chart where $X \neq 0$ defines a `complex' cusp by analogy with $\mathbb{R}^2$ (therefore derivative not invertible at the cusp).
However, using the inverse function theorem, if $\nabla F \neq 0$ on the zeros locus, then it defines a submanifold of $\mathbb{C}P^n$. 
This is not obvious as we start with coordinates on $\mathbb{C}^{n+1}$, meaning that we have a submanifold of $\mathbb{C}^{n+1}$ and not $\mathbb{C}P^n$; some care is required.
% 
\definition{
    For a collections of homogeneous polynomials. The simultaneous zero locus is called an \textit{algebraic set}. If it irreducible and connected, it is called a \textit{projective variety}. But note that not all projective variety are manifold.
}

% 
So we have a way to construct compact complex submanifold of $\mathbb{C}P^n$ by taking $r < n$ homogeneous polynomials, then given a simultaneous null regular value (i.e: $\nabla f|_p \neq 0$ with $p=0$) defines a complex an $(n-r)$-compact complex submanifold.
Even though we have generated lot of compact complex manifold, they are \textbf{not} all projective.
\begin{itemize}
    \item Q: Which compact complex manifold can you embed into $\mathbb{C}P^n$?
    \item Q: Which of them are projective varieties?
\end{itemize}

\theorem{
     The Kodaira embedding theorem gives an if and only if criterion for Q1.
     basically you are in projective space if you admit a nice holomorphic line bundle.
}
% subsection projective_varieties (end)
% --------------------------------------------------------------------------------------------
% --------------------------------------------------------------------------------------------
% --------------------------------------------------------------------------------------------
