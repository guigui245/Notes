% !TeX root = ../main.tex
% 
\lhead{\emph{Lecture 4}}
\section{Lecture 4} % (fold)
\label{sec:lecture_4}
talk abt other calabi conjectures, no proofs for that.
last time we defined kahler mfd as compact complex with compatible riemanian metric, closed $1,1$ form
or holomorphic normal coordinates around every points. not every complex mfd have kahler metric
More general, if have holomorphic line bundle, we can define kahler metric, using chern connection and chern class. closed globally from $\Theta$. defines a well-defined cohomology class 
\question{given a positive top form, is it a volume? 1st calabi conjecture. Gives pde}

\definition{
    Cananical bundle of $\mathbb{C}P^n$. if M complex manifold, then holomorphic kine bundle $\Omega^{n,0}$ is canonical $K_M$ the top exterior forms is the determinant.
    Its dual is called the anti-canonical bundle.
}
\remark{
if $dim_\mathbb{C}(M) = 1$ these are riemann surfaces, they are all projective and $K_M$ is the cotangent bundle.
    
}
Since we hace explicit formula for fubini-study form
\begin{equation}
    \omega_{FS} = i \partial \overline{\partial} \ln(1 + \ldots)
\end{equation}

compute ites first chern form
\begin{equation}
    c_1(K_S, \omega_{FS}) = \frac{i}{2 \pi} \partial \overline{\partial} \ln \det(\omega_{FS})
\end{equation}
find $c_1(K_S) = -(n+1)\omega_{FS}$ for $\mathbb{C}P^n$

\question{if the first chern class is as above, can we conclude that the mfd is $\mathbb{C}P^n?$ open question
}

canonical bundle of hypersurface: given $F$ a degree d poly, then $\mathrm{ker}(F)$ definces $S =\in \mathbb{C}P^n$ and restrict metric to
\begin{equation}
    \omega_M= i^*\omega_{FS}
\end{equation}
hard calculation is hard but
\begin{equation}
    c_1(K_S) = (d-n-1)w_{FS} + i \partial \overline{\partial} \psi
\end{equation}
look at bando-futaki invations on hypersurfaces by liu.
Can be proved by adjounction formulat

now recall $c_1(S) = -c_1(K_S) = (n+1-d)\omega_FS - i \partial \overline{\partial} \psi$
let $h = det(\omega_{FS})e^{2 \pi \psi}$ a new metric on $K_S^*$ then
\begin{equation}
    c_1(K_S^*,h) = c_1(K_S) - i \partial \overline{\partial} \psi
\end{equation}

sp the anti canonical line bundle is a positive line budle for $d < n+1$ and negative for less. For $d = n+1$ then $c_1(S) = [0]$. special name Calabi yau manifold

For example the fermat quintic
\begin{equation}
    \sum_{i=0}^5 (X^i)^5 = 0
\end{equation}
in $\mathbb{C}P^4$ has $c_1 = [0]$ this is a calabi yau $3$-manifold.
\subsection{connection on kahler mfd} % (fold)
\label{sub:connection_on_kahler_mfd}
Let (M,h) a complex manifold with hermitian metric h; it induces a compatible Riemannian metric.
there are 2 connections on $T^{0,1}M$, the levi-civita of g and the chern connectiuon of $h$.
\proposition{
    If M is kahler, these 2 coincide!
}
\begin{proof}
    Choose holomorphic normal coordinates around $p$. all metrics are standard up to $1$st order.
    Therefore christofell symbols for both metrics are 0 at this point.
    The converse is true but this shortcut does not work.
    If they coincide on one patch, then by holomorphic can extend.
\end{proof}
can link algebraic goemetry and riemann geometry in the case of kahler.

As a consequence, the curvature tensor of the kahler metric has the same symmetry as riemann connection.

\definition{
    Riemannian tensor
    \begin{equation}
        R(X,Y)Z = \Theta^i_{j}Z^j (X,Y) ?
    \end{equation}
}
\definition{
    Ricci tensor
    \begin{equation}
        Ric(Y,Z) = \mathrm{tr}\left(X \rightarrow R(X,Y)Z\right)
    \end{equation}
    so $Ric_{kl} = \Theta^i_j(\partial_i, \partial_l)$
}
so in holomorphic coordinates
qe have
\begin{equation}
    \Theta^i_j = \overline{\partial} \partial \overline{h}_{ij} = \overline{\partial}\partial 
    \frac{\partial^2 \phi}{\partial \overline{z}^i \partial z^j}
\end{equation}

so in these coordinates
\begin{equation}
    Ric_{kl} = \sum \frac{\partial^4 \phi}{\partial \partial \partial \partial}
\end{equation}
hence
\begin{equation}
    Ric_{kl}dz^k \wedge d \overline{z}^l = \mathrm{tr}(\Theta)(\partial_{\overline{l}}, \partial_k)
\end{equation}
the normalised ricci form is thus $C_1(M)$.
% subsection connection_on_kahler_mfd (end)
\subsection{another calabi conjecture} % (fold)
\label{sub:another_calabi_conjecture}
Given a real $d$-closed $1,1$-form $\eta$ such that $[\eta] = c_1(M)$ and kahler class $[\Omega]$, ther exists a unique kahler metric
\begin{equation}
    \omega_\phi = \omega + i \iota \partial \overline{\partial} \phi
\end{equation}
such that $Ric(\omega_\phi) = \eta$.
Solve by Yau.
When $c_1(M) = [0]$, there exists a ricci flat metric. so there is a ricci flat metric on the fermat quintic $Q_5$.
But if the metric is flat then one can prove the manifold is isomorphic to a torus!

However, one can show $Q_5$ is simply connected using morse theory. therefore it cannot be a torus! so it is ricci flat but with non-flat metric. first interesting example of a calabi-yau manifold.

Moreover, if $c_1(M) >0$ then it has a kahler form. Then can choose $\eta$ to be a positive form. In other words, calabi conjecture implies there is a positive ricci metric, the ricci curvature of such metric is positive.

Recall, if ricci curvature is positive of compact manifold, by Bonnet-Myers theorem, the fundamental group of M is finite.
More details: bonnet myers: if ricci curvature is bounded below, then diameter is bounded above, then manifold is compact.
So if ricci curvature bounded below, then pullback to universal cover is also compact so fundamental group is finite.
but can check using algebraic geometry that $c_1(M)$ to look at topology. link here
Such manifold are called fano manifolds (when simply connected?)
% subsection another_calabi_conjecture (end)
\subsection{reduction to simpler pde} % (fold)
\label{sub:reduction_to_simpler_pde}
a priori this is a 4th order pde.
but we note
\begin{equation}
    \eta - Ric(\omega) = i \partial \overline{\partial} f
\end{equation}
but we want $Ric(\omega_\phi) = Ric(\omega) + i \partial \overline{\partial} f$.
Hence
\begin{equation}
    i \partial \overline{\partial} \ln(\frac{\omega^n}{\omega^n_\phi}) = i \partial \overline{\partial} f
\end{equation}
hence difference of these two is a cte.
therefore
\begin{equation}
    \omega_\phi^n = e^{\tilde{f} }\omega^n
\end{equation}
where 
\begin{equation}
    \int \omega^n (e^{\tilde{f}}-1) = 0
\end{equation}
this equation is the monge ampere equation encountred in the other calabi conjecture.
% subsection reduction_to_simpler_pde (end)
\subsection{Kahler-einstein metrics} % (fold)
\label{sub:kahler_einstein_metrics}

Form gr,
\begin{equation}
    Ric(g) = \lambda g + \rho
\end{equation}
where $\rho$ has the matter energy density. $\lambda$ is related to cosmological constant.
In vaccum, einstein metrics $Ric(g) = \lambda g$

A KE metric $\omega$ is one that satisfies $Ric(\omega) = \lambda \omega$, so 
$c_1(M)$ has sign of $\lambda$.

\conjecture{
    if $C_1(M)$ has a sign and $|\lambda| =1$ or 0, then there is a unique KE metric.
}

This is proved by Yau for $c_1 <0$ but it is \textbf{False} for $c_1 >0$ in general,(the fano manifolds).
(btw all orientable surfaces have a complex structure)
A priori this is also a 4th order pde. but can be reduced also.

let $|\lambda| = 1$
\begin{equation}
    \lambda \omega - Ric(\omega) = i \partial \overline{\partial} f
\end{equation}
hence go to $\omega_\phi$
thus
\begin{equation}
    i \partial \overline{\partial} \ln(\frac{\omega^n}{\omega^n_\phi}) = i \partial \overline{\partial} (\lambda \phi + f)
\end{equation}
normalise
$\omega_\phi^n = e^{-f - \lambda \phi} \omega^n $, similar to previous eqution, but behaves very differenlty

If $\lambda = -1$, uniqueness is easy.
Spose $\omega, \omega_\phi$ are 2 KE metrics, then $f =0$ hence
$\omega_f^n = e^\phi \omega^n$, at maximum of $\phi$, we have $\omega_\phi \leq \omega^n$. Hence $\phi \leq 0$ likewise for positive, hence $\phi = 0$
% subsection kahler_einstein_metrics (end)
\subsection{constant scalar curvature} % (fold)
\label{sub:constant_scalar_curvature}
The point of KE metrics is algebraic geometry.
If there is a KE metric, then there areinequalities between chern classes.
If $c_1<0$, there are inequalities.
One can construct moduli spaces of manifold that mantain KE metric in an analytical manner. ??dafuq
If $[\omega]$ no a multiple of $c_1(M)$, can we find a metric in that class?
Such classes can be given by $c_1(L)$ for some line L.
Could parametrics pairs of line bundles $(M,L)$ suing classes.

cscK metric
Scalar curvature
\begin{equation}
    Ric_{ij} g^{ij} =? \frac{n \omega^{n-1}c_1(\omega)}{\omega^n}
\end{equation}
exercise
\conjecture{
    Existence and uniqueness of cscK metrics? Question still open in general
}

KE metrics are cscK metrics.

parametrising triples with vector bundles $(M,L,E)$, get Kahler-yang-mills equations.

\begin{proof}
    sketch of calabi yau conjecture.
    Method of continuity, family of pde
    \begin{equation}
        \omega^n_{\phi_t} = e^{tf}\omega^n
    \end{equation}
    at $t=0$, the obvious somlution.
    Want to prove that the set of $t \in [0,1]$ solving the equation is open and closd.
    For openness, one uses infinite finite-dimensional implicit function theorem, to reduce to the solvability of an linear elliptic pde. (laplace like equation)\\
    to prove closedness, suppose $t_n \rightarrow t_0$ $t_n$ a colleciton of times that are solutions.
    need to show that 
    \begin{equation}
        \norm{\phi}_{C^{2,\alpha}} \leq C
    \end{equation}
    for all $t_n< t_0$. $\alpha$ is from the ``heldard norm?''
    using the Arzela-ascoli theorem, a subsequence in $C^{2,\beta}$ $\beta < \alpha$ to a solution. Smoothness follows from linear elliptic pde.
    The $C^0$ estimate is the hardest part and is the one that fials for $c_1 >0$ KE case.
\end{proof}


% subsection constant_scalar_curvature (end)
% section lecture_4 (end)